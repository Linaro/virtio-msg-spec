\section{CAN Device}\label{sec:Device Types / CAN Device}

virtio-can is a virtio based CAN (Controller Area Network) controller.
It is used to give a virtual machine access to a CAN bus. The CAN bus
might either be a physical CAN bus or a virtual CAN bus between virtual
machines or a combination of both.

\subsection{Device ID}\label{sec:Device Types / CAN Device / Device ID}

36

\subsection{Virtqueues}\label{sec:Device Types / CAN Device / Virtqueues}

\begin{description}
\item[0] Txq
\item[1] Rxq
\item[2] Controlq
\end{description}

The \field{Txq} is used to send CAN packets to the CAN bus.

The \field{Rxq} is used to receive CAN packets from the CAN bus.

The \field{Controlq} is used to control the state of the CAN controller.

\subsection{Feature bits}{Device Types / CAN Device / Feature bits}

Actual CAN controllers support Extended CAN IDs with 29 bits (CAN~2.0B)
as well as Standard CAN IDs with 11 bits (CAN~2.0A). The support of
CAN~2.0B Extended CAN IDs is considered as mandatory for this
specification.

\begin{description}

\item[VIRTIO_CAN_F_CAN_CLASSIC (0)]

The device supports classic CAN frames with a maximum payload size of 8
bytes.

\item[VIRTIO_CAN_F_CAN_FD (1)]

The device supports CAN FD frames with a maximum payload size of 64
bytes.

\item[VIRTIO_CAN_F_RTR_FRAMES (2)]

The device supports RTR (remote transmission request) frames. RTR frames
are only supported with classic CAN.

\item[VIRTIO_CAN_F_LATE_TX_ACK (3)]

The virtio CAN device marks transmission requests from the \field{Txq}
as used after the CAN message has been transmitted on the CAN bus. If
this feature bit has not been negotiated, the device is allowed to mark
transmission requests already as used when the CAN message has been
scheduled for transmission but might not yet have been transmitted on
the CAN bus.

\end{description}

\subsubsection{Feature bit requirements}\label{sec:Device Types / CAN Device / Feature bits / Feature bit requirements}

Some CAN feature bits require other CAN feature bits:
\begin{description}
\item[VIRTIO_CAN_F_RTR_FRAMES] Requires VIRTIO_CAN_F_CAN_CLASSIC.
\end{description}

It is required that at least one of VIRTIO_CAN_F_CAN_CLASSIC and
VIRTIO_CAN_F_CAN_FD is negotiated.

\subsection{Device configuration layout}\label{sec:Device Types / CAN Device / Device configuration layout} 

Device configuration fields are listed below, they are read-only for a
driver. The \field{status} always exists. A single read-only bit (for
the driver) is currently defined for \field{status}:

\begin{lstlisting}
struct virtio_can_config {
#define VIRTIO_CAN_S_CTRL_BUSOFF (1 << 0)
        le16 status;
};
\end{lstlisting}

The bit VIRTIO_CAN_S_CTRL_BUSOFF in \field{status} is used to indicate
the unsolicited CAN controller state change from started to stopped due
to a detected bus off condition.

\drivernormative{\subsubsection}{Device Initialization}{Device Types / CAN Device / Device Operation / Initialization}

The driver MUST populate the \field{Rxq} with empty device-writeable
buffers of at least the size of struct virtio_can_rx, see section
\ref{struct virtio-can-rx}.

\subsection{Device Operation}\label{sec:Device Types / CAN Device / Device Operation}

A device operation has an outcome which is described by one of the
following values:

\begin{lstlisting}
#define VIRTIO_CAN_RESULT_OK     0
#define VIRTIO_CAN_RESULT_NOT_OK 1
\end{lstlisting}

Other values are to be treated like VIRTIO_CAN_RESULT_NOT_OK.

\subsubsection{Controller Mode}\label{sec:Device Types / CAN Device / Device Operation / Controller Mode}

The general format of a request in the \field{Controlq} is

\begin{lstlisting}
struct virtio_can_control_out {
#define VIRTIO_CAN_SET_CTRL_MODE_START  0x0201
#define VIRTIO_CAN_SET_CTRL_MODE_STOP   0x0202
        le16 msg_type; 
};
\end{lstlisting}

To participate in bus communication the CAN controller is started by
sending a VIRTIO_CAN_SET_CTRL_MODE_START control message, to stop
participating in bus communication it is stopped by sending a
VIRTIO_CAN_SET_CTRL_MODE_STOP control message. Both requests are
confirmed by the result of the operation.

\begin{lstlisting}
struct virtio_can_control_in {
        u8 result;
};
\end{lstlisting}

If the transition succeeded the \field{result} is VIRTIO_CAN_RESULT_OK
otherwise it is VIRTIO_CAN_RESULT_NOT_OK. If a status update is
necessary, the device updates the configuration \field{status} before
marking the request used. As the configuration \field{status} change is
caused by a request from the driver the device is allowed to omit the
configuration change notification here. The device marks the request
used when the CAN controller has finalized the transition to the
requested controller mode.

On transition to the STOPPED state the device cancels all CAN messages
already pending for transmission and marks them as used with
\field{result} VIRTIO_CAN_RESULT_NOT_OK. In the STOPPED state the
device marks messages received from the
\field{Txq} as used with \field{result} VIRTIO_CAN_RESULT_NOT_OK without
transmitting them to the CAN bus.

Initially the CAN controller is in the STOPPED state.

Control queue messages are processed in order.

\devicenormative{\subsubsection}{CAN Message Transmission}{Device Types / CAN Device / Device Operation / CAN Message Transmission}

The driver transmits messages by placing outgoing CAN messages in the
\field{Txq} virtqueue.

\label{struct virtio-can-tx-out}
\begin{lstlisting}
struct virtio_can_tx_out {
#define VIRTIO_CAN_TX 0x0001
        le16 msg_type;
        le16 length; /* 0..8 CC, 0..64 CAN-FD, 0..2048 CAN-XL, 12 bits */
        u8 reserved_classic_dlc; /* If CAN classic length = 8 then DLC can be 8..15 */
        u8 padding;
        le16 reserved_xl_priority; /* May be needed for CAN XL priority */
#define VIRTIO_CAN_FLAGS_FD            0x4000
#define VIRTIO_CAN_FLAGS_EXTENDED      0x8000
#define VIRTIO_CAN_FLAGS_RTR           0x2000
        le32 flags;
        le32 can_id;
        u8 sdu[];
};

struct virtio_can_tx_in {
        u8 result;
};
\end{lstlisting}

The length of the \field{sdu} is determined by the \field{length}.

The type of a CAN message identifier is determined by \field{flags}. The
3 most significant bits of \field{can_id} do not bear the information
about the type of the CAN message identifier and are 0.

The device MUST reject any CAN frame type for which support has not been
negotiated with VIRTIO_CAN_RESULT_NOT_OK in \field{result} and MUST NOT
schedule the message for transmission. A CAN frame with an undefined bit
set in \field{flags} is treated like a CAN frame for which support has
not been negotiated.

The device MUST reject any CAN frame for which \field{can_id} or
\field{sdu} length are out of range or the CAN controller is in an
invalid state with VIRTIO_CAN_RESULT_NOT_OK in \field{result} and MUST
NOT schedule the message for transmission.

If the parameters are valid the message is scheduled for transmission.

If feature VIRTIO_CAN_F_CAN_LATE_TX_ACK has been negotiated the
transmission request MUST be marked as used with \field{result} set to
VIRTIO_CAN_OK after the CAN controller acknowledged the successful
transmission on the CAN bus. If this feature bit has not been negotiated
the transmission request MAY already be marked as used with
\field{result} set to VIRTIO_CAN_OK when the transmission request has
been processed by the virtio CAN device and send down the protocol stack
being scheduled for transmission.

\subsubsection{CAN Message Reception}\label{sec:Device Types / CAN Device / Device Operation / CAN Message Reception}

Messages can be received by providing empty incoming buffers to the
virtqueue \field{Rxq}.

\label{struct virtio-can-rx}
\begin{lstlisting}
struct virtio_can_rx {
#define VIRTIO_CAN_RX 0x0101
        le16 msg_type;
        le16 length; /* 0..8 CC, 0..64 CAN-FD, 0..2048 CAN-XL, 12 bits */
        u8 reserved_classic_dlc; /* If CAN classic length = 8 then DLC can be 8..15 */
        u8 padding;
        le16 reserved_xl_priority; /* May be needed for CAN XL priority */
        le32 flags;
        le32 can_id;
        u8 sdu[];
};
\end{lstlisting}

If the feature VIRTIO_CAN_F_CAN_FD has been negotiated the maximal
possible \field{sdu} length is 64, if the feature has not been
negotiated the maximal possible \field{sdu} length is 8.

The actual length of the \field{sdu} is determined by the \field{length}.

The type of a CAN message identifier is determined by \field{flags} in
the same way as for transmitted CAN messages, see section
\ref{struct virtio-can-tx-out}. The 3 most significant bits of \field{can_id} do not
bear the information about the type of the CAN message identifier and
are 0. The flag bits are exactly the same as for the \field{flags} of
struct virtio_can_tx_out.

\subsubsection{BusOff Indication}\label{sec:Device Types / CAN Device / Device Operation / BusOff Indication}

There are certain error conditions so that the physical CAN controller
has to stop participating in CAN communication on the bus. If such an
error condition occurs the device informs the driver about the
unsolicited CAN controller state change by setting the
VIRTIO_CAN_S_CTRL_BUSOFF bit in the configuration \field{status} field.

After bus-off detection the CAN controller is in STOPPED state. The CAN
controller does not participate in bus communication any more so all CAN
messages pending for transmission are put into the used queue with
\field{result} VIRTIO_CAN_RESULT_NOT_OK.
