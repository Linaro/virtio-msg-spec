\section{SPI Controller Device}\label{sec:Device Types / SPI Controller Device}

The Virtio SPI (Serial Peripheral Interface) device is a virtual SPI controller that
allows the driver to operate and use the SPI controller under the control of the host,
either a physical SPI controller, or an emulated one.

The Virtio SPI device has a single virtqueue. SPI transfer requests are placed into
the virtqueue by the driver, and are serviced by the device.

In a typical host and guest architecture with the Virtio SPI device, the Virtio SPI driver
is the front-end running in the guest, and the Virtio SPI device is the back-end
in the host.

\subsection{Device ID}\label{sec:Device Types / SPI Controller Device / Device ID}
45

\subsection{Virtqueues}\label{sec:Device Types / SPI Controller Device / Virtqueues}

\begin{description}
\item[0] requestq
\end{description}

\subsection{Feature bits}\label{sec:Device Types / SPI Controller Device / Feature bits}

None

\subsection{Device configuration layout}\label{sec:Device Types / SPI Controller Device / Device configuration layout}

All fields of this configuration are mandatory and read-only for the driver.
The config space shows the features and settings supported by the device.

\begin{lstlisting}
struct virtio_spi_config {
	u8 cs_max_number;
	u8 cs_change_supported;
	u8 tx_nbits_supported;
	u8 rx_nbits_supported;
	le32 bits_per_word_mask;
	le32 mode_func_supported;
	le32 max_freq_hz;
	le32 max_word_delay_ns;
	le32 max_cs_setup_ns;
	le32 max_cs_hold_ns;
	le32 max_cs_inactive_ns;
};
\end{lstlisting}

\field{cs_max_number} is the maximum number of chipselect the device supports.

Note: chipselect is an electrical signal, which is used to select the SPI peripherals connected
to the controller.

\field{cs_change_supported} indicates if the device supports to toggle chipselect
after each transfer in one message:
        0: unsupported, chipselect will be kept in active state throughout the message transaction;
        1: supported.

Note: Message here contains a sequence of SPI transfers.

\field{tx_nbits_supported} and \field{rx_nbits_supported} indicate the different n-bit transfer
modes supported by the device, for writing and reading respectively. SINGLE is always supported.
A set bit here indicates that the corresponding n-bit transfer is supported, otherwise not:
        bit 0: DUAL;
        bit 1: QUAD;
        bit 2: OCTAL;
        other bits are reserved and must be set as 0 by the device.

Note: The commonly used SPI n-bit transfer options are:
\begin{itemize}
\item SINGLE: 1-bit transfer
\item DUAL: 2-bit transfer
\item QUAD: 4-bit transfer
\item OCTAL: 8-bit transfer
\end{itemize}

\field{bits_per_word_mask} is a mask indicating which values of bits_per_word are supported.
If bit n of \field{bits_per_word_mask} is set, the bits_per_word with value (n+1) is supported.
If \field{bits_per_word_mask} is 0, there is no limitation for bits_per_word.

Note: bits_per_word corresponds to \field{bits_per_word} in \field{struct virtio_spi_transfer_head}.

\field{mode_func_supported} indicates whether the following features are supported or not:
        bit 0-1: CPHA feature,
            0b00: invalid, must support as least one CPHA setting.
            0b01: supports CPHA=0 only;
            0b10: supports CPHA=1 only;
            0b11: supports CPHA=0 and CPHA=1;

        bit 2-3: CPOL feature,
            0b00: invalid, must support as least one CPOL setting.
            0b01: supports CPOL=0 only;
            0b10: supports CPOL=1 only;
            0b11: supports CPOL=0 and CPOL=1;

        bit 4: chipselect active high feature, 0 for unsupported and 1 for supported,
            chipselect active low must always be supported.

        bit 5: LSB first feature, 0 for unsupported and 1 for supported, MSB first must always be
            supported.

        bit 6: loopback mode feature, 0 for unsupported and 1 for supported, normal mode
            must always be supported.

Note: CPOL is clock polarity and CPHA is clock phase. If CPOL is 0, the clock idles at the logical
low voltage, otherwise it idles at the logical high voltage. CPHA determines how data is outputted
and sampled. If CPHA is 0, the first bit is outputted immediately when chipselect is active and
subsequent bits are outputted on the clock edges when the clock transitions from active level to idle
level. Data is sampled on the clock edges when the clock transitions from idle level to active level. 
If CPHA is 1, the first bit is outputted on the fist clock edge after chipselect is active, subsequent
bits are outputted on the clock edges when the clock transitions from idle level to active level.
Data is sampled on the clock edges when the clock transitions from active level to idle level.

Note: LSB first indicates that data is transferred least significant bit first, and MSB first
indicates that data is transferred most significant bit first.

\field{max_freq_hz} is the maximum clock rate supported in Hz unit, 0 means no limitation
for transfer speed.

\field{max_word_delay_ns} is the maximum word delay supported in ns unit, 0 means word delay
feature is unsupported.

Note: Just as one message contains a sequence of transfers, one transfer may contain
a sequence of words.

\field{max_cs_setup_ns} is the maximum delay supported after chipselect is asserted, in ns unit,
0 means delay is not supported to introduce after chipselect is asserted.

\field{max_cs_hold_ns} is the maximum delay supported before chipselect is deasserted, in ns unit,
0 means delay is not supported to introduce before chipselect is deasserted.

\field{max_cs_incative_ns} is the maximum delay supported after chipselect is deasserted, in ns unit,
0 means delay is not supported to introduce after chipselect is deasserted.

\subsection{Device Initialization}\label{sec:Device Types / SPI Controller Device / Device Initialization}

\begin{enumerate}
\item The Virtio SPI driver configures and initializes the virtqueue.
\end{enumerate}

\subsection{Device Operation}\label{sec:Device Types / SPI Controller Device / Device Operation}

\subsubsection{Device Operation: Request Queue}\label{sec:Device Types / SPI Controller Device / Device Operation: Request Queue}

The Virtio SPI driver enqueues requests to the virtqueue, and they are used by the Virtio SPI device.
Each request represents one SPI transfer and is of the form:

\begin{lstlisting}
struct virtio_spi_transfer_head {
        u8 chip_select_id;
        u8 bits_per_word;
        u8 cs_change;
        u8 tx_nbits;
        u8 rx_nbits;
        u8 reserved[3];
        le32 mode;
        le32 freq;
        le32 word_delay_ns;
        le32 cs_setup_ns;
        le32 cs_delay_hold_ns;
        le32 cs_change_delay_inactive_ns;
};
\end{lstlisting}

\begin{lstlisting}
struct virtio_spi_transfer_result {
        u8 result;
};
\end{lstlisting}

\begin{lstlisting}
struct virtio_spi_transfer_req {
        struct virtio_spi_transfer_head head;
        u8 tx_buf[];
        u8 rx_buf[];
        struct virtio_spi_transfer_result result;
};
\end{lstlisting}

\field{chip_select_id} indicates the chipselect index to use for the SPI transfer.

\field{bits_per_word} indicates the number of bits in each SPI transfer word.

\field{cs_change} indicates whether to deselect device before starting the next SPI transfer,
0 means chipselect keep asserted and 1 means chipselect deasserted then asserted again.

\field{tx_nbits} and \field{rx_nbits} indicate n-bit transfer mode for writing and reading:
        0,1: SINGLE;
        2  : DUAL;
        4  : QUAD;
        8  : OCTAL;
        other values are invalid.

\field{reserved} is currently unused and might be used for further extensions in the future.

\field{mode} indicates some transfer settings. Bit definitions as follows:
        bit 0: CPHA, determines the timing (i.e. phase) of the data bits relative to the
	       clock pulses.
        bit 1: CPOL, determines the polarity of the clock.
        bit 2: CS_HIGH, if 1, chipselect active high, else active low.
        bit 3: LSB_FIRST, determines per-word bits-on-wire, if 0, MSB first, else LSB first.
        bit 4: LOOP, if 1, device is in loopback mode, else normal mode.

\field{freq} indicates the SPI transfer speed in Hz.

\field{word_delay_ns} indicates delay to be inserted between consecutive words of a transfer,
in ns unit.

\field{cs_setup_ns} indicates delay to be introduced after chipselect is asserted, in ns unit.

\field{cs_delay_hold_ns} indicates delay to be introduced before chipselect is deasserted,
in ns unit.

\field{cs_change_delay_inactive_ns} indicates delay to be introduced after chipselect is
deasserted and before next asserted, in ns unit.

\field{tx_buf} is the buffer for data sent to the device.

\field{rx_buf} is the buffer for data received from the device.

\field{result} is the transfer result, it may be one of the following values:

\begin{lstlisting}
#define VIRTIO_SPI_TRANS_OK     0
#define VIRTIO_SPI_PARAM_ERR    1
#define VIRTIO_SPI_TRANS_ERR    2
\end{lstlisting}

VIRTIO_SPI_TRANS_OK indicates successful completion of the transfer.

VIRTIO_SPI_PARAM_ERR indicates a parameter error, which means the parameters in
\field{struct virtio_spi_transfer_head} are not all valid, or some fields are set as
non-zero values but the corresponding features are not supported by device.
In particular, for full-duplex transfer, VIRTIO_SPI_PARAM_ERR can also indicate that
\field{tx_buf} and \field{rx_buf} are not of the same length.

VIRTIO_SPI_TRANS_ERR indicates a transfer error, which means that the parameters are all
valid but the transfer process failed.

\subsubsection{Device Operation: Operation Status}\label{sec:Device Types / SPI Controller Device / Device Operation: Operation Status}

Fields in \field{struct virtio_spi_transfer_head} are written by the Virtio SPI driver, while
\field{result} in \field{struct virtio_spi_transfer_result} is written by the Virtio SPI device.

virtio-spi supports three transfer types:
\begin{itemize}
\item half-duplex read;
\item half-duplex write;
\item full-duplex transfer.
\end{itemize}

For half-duplex read and full-duplex transfer, \field{rx_buf} is filled by the Virtio SPI device
and consumed by the Virtio SPI driver. For half-duplex write and full-duplex transfer, \field{tx_buf}
is filled by the Virtio SPI driver and consumed by the Virtio SPI device.

\drivernormative{\subsubsection}{Device Operation}{Device Types / SPI Controller Device / Device Operation}

For half-duplex read, the Virtio SPI driver MUST send \field{struct virtio_spi_transfer_head},
\field{rx_buf} and \field{struct virtio_spi_transfer_result} to the SPI Virtio Device in that order.

For half-duplex write, the Virtio SPI driver MUST send \field{struct virtio_spi_transfer_head},
\field{tx_buf} and \field{struct virtio_spi_transfer_result} to the SPI Virtio Device in that order.

For full-duplex transfer, the Virtio SPI driver MUST send \field{struct virtio_spi_transfer_head},
\field{tx_buf}, \field{rx_buf} and \field{struct virtio_spi_transfer_result} to the SPI Virtio Device
in that order.

For full-duplex transfer, the Virtio SPI driver MUST guarantee that the buffer size of \field{tx_buf}
and \field{rx_buf} is the same.

The Virtio SPI driver MUST not use \field{rx_buf} if the \field{result} returned from the Virtio SPI device is
not VIRTIO_SPI_TRANS_OK.

\devicenormative{\subsubsection}{Device Operation}{Device Types / SPI Controller Device / Device Operation}

The Virtio SPI device MUST set all the fields of \field{struct virtio_spi_config} before they are
read by the Virtio SPI driver.

The Virtio SPI device MUST NOT change the data in \field{tx_buf}.

The Virtio SPI device MUST verify the parameters in \field{struct virtio_spi_transfer_head} after receiving
the request, and MUST set \field{struct virtio_spi_transfer_result} as VIRTIO_SPI_PARAM_ERR if not all
parameters are valid or some device unsupported features are set.

For full-duplex transfer, the Virtio SPI device MUST verify that the buffer size of \field{tx_buf} is equal to
that of \field{rx_buf}. If not, the Virtio SPI device MUST set \field{struct virtio_spi_transfer_result}
as VIRTIO_SPI_PARAM_ERR.
