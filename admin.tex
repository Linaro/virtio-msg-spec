\section{Device groups}\label{sec:Basic Facilities of a Virtio Device / Device groups}

It is occasionally useful to have a device control a group of
other devices (the group may occasionally include the device
itself) within a group. The owner device itself is not a
member of the group (except in the special case of the self group).
Terminology used in such cases:

\begin{description}
\item[Device group]
        or just group, includes zero or more devices.
\item[Owner device]
        or owner, the device controlling the group.
\item[Member device]
        a device within a group. The owner device itself is not
	a member of the group except for the \field{Self group type}.
\item[Member identifier]
        each member has this identifier, unique within the group
	and used to address it through the owner device.
\item[Group type identifier]
	specifies what kind of member devices there are in a
	group, how the member identifier is interpreted
	and what kind of control the owner has.
	A given owner can control multiple groups
	of different types but only a single group of a given type,
	thus the type and the owner together identify the group.
	\footnote{Even though some group types only support
			specific transports, group type identifiers
			are global rather than transport-specific -
			a flood of new group types is not expected.}
\end{description}

\begin{note}
Each device only has a single driver, thus for the purposes of
this section, "the driver" is usually unambiguous and refers to
the driver of the owner device.  When there's ambiguity, "owner
driver" refers to the driver of the owner device, while "member
driver" refers to the driver of a member device.
\end{note}

The following group types, and their identifiers, are currently specified:
\begin{description}
\item[Self group type (0x0)]
This device group includes the owner device itself and no other devices.
The group type identifier for this group is 0x0.
The member identifier for this group has a value of 0x0.

\item[SR-IOV group type (0x1)]
This device group has a PCI Single Root I/O Virtualization
(SR-IOV) physical function (PF) device as the owner and includes
all its SR-IOV virtual functions (VFs) as members (see
\hyperref[intro:PCIe]{[PCIe]}).

The PF device itself is not a member of the group.

The group type identifier for this group is 0x1.

A member identifier for this group can have a value from 0x1 to
\field{NumVFs} as specified in the
SR-IOV Extended Capability of the owner device
and equals the SR-IOV VF number of the member device;
the group only exists when the \field{VF Enable} bit
in the SR-IOV Control Register within the
SR-IOV Extended Capability of the owner device is set
(see \hyperref[intro:PCIe]{[PCIe]}).

Both owner and member devices for this group type use the Virtio
PCI transport (see \ref{sec:Virtio Transport Options / Virtio Over PCI Bus}).
\end{description}

\subsection{Group administration commands}\label{sec:Basic Facilities of a Virtio Device / Device groups / Group administration commands}

The driver sends group administration commands to the owner device of
a group to control member devices of the group.
This mechanism can
be used, for example, to configure a member device before it is
initialized by its driver.
\footnote{The term "administration" is intended to be interpreted
widely to include any kind of control. See specific commands
for detail.}

All the group administration commands are of the following form:

\begin{lstlisting}
struct virtio_admin_cmd {
        /* Device-readable part */
        le16 opcode;
        /*
         * 0       - Self
         * 1       - SR-IOV
         * 2-65535 - reserved
         */
        le16 group_type;
        /* unused, reserved for future extensions */
        u8 reserved1[12];
        le64 group_member_id;
        le64 command_specific_data[];

        /* Device-writable part */
        le16 status;
        le16 status_qualifier;
        /* unused, reserved for future extensions */
        u8 reserved2[4];
        u8 command_specific_result[];
};
\end{lstlisting}

For all commands, \field{opcode}, \field{group_type} and if
necessary \field{group_member_id} and \field{command_specific_data} are
set by the driver, and the owner device sets \field{status} and if
needed \field{status_qualifier} and
\field{command_specific_result}.

Generally, any unused device-readable fields are set to zero by the driver
and ignored by the device.  Any unused device-writeable fields are set to zero
by the device and ignored by the driver.

\field{opcode} specifies the command. The valid
values for \field{opcode} can be found in the following table:

\begin{xltabular}{\textwidth}{ |l||l|X| }
\hline
opcode & Name & Command Description \\
\hline \hline
0x0000 & VIRTIO_ADMIN_CMD_LIST_QUERY & Provides to driver list of commands supported for this group type    \\
\hline
0x0001 & VIRTIO_ADMIN_CMD_LIST_USE & Provides to device list of commands used for this group type \\
\hline
0x0002 & \hyperref[par:Basic Facilities of a Virtio Device / Device groups / Group administration commands / Legacy Interface / VIRTIO_ADMIN_CMD_LEGACY_COMMON_CFG_WRITE]{VIRTIO_ADMIN_CMD_LEGACY_COMMON_CFG_WRITE} & Writes into the legacy common configuration structure \\
\hline
0x0003 & \hyperref[par:Basic Facilities of a Virtio Device / Device groups / Group administration commands / Legacy Interface / VIRTIO_ADMIN_CMD_LEGACY_COMMON_CFG_READ]{VIRTIO_ADMIN_CMD_LEGACY_COMMON_CFG_READ} & Reads from the legacy common configuration structure  \\
\hline
0x0004 & \hyperref[par:Basic Facilities of a Virtio Device / Device groups / Group administration commands / Legacy Interface / VIRTIO_ADMIN_CMD_LEGACY_DEV_CFG_WRITE]{VIRTIO_ADMIN_CMD_LEGACY_DEV_CFG_WRITE} & Writes into the legacy device configuration structure \\
\hline
0x0005 & \hyperref[par:Basic Facilities of a Virtio Device / Device groups / Group administration commands / Legacy Interface / VIRTIO_ADMIN_CMD_LEGACY_DEV_CFG_READ]{VIRTIO_ADMIN_CMD_LEGACY_DEV_CFG_READ} & Reads into the legacy device configuration structure \\
\hline
0x0006 & \hyperref[par:Basic Facilities of a Virtio Device / Device groups / Group administration commands / Legacy Interface / VIRTIO_ADMIN_CMD_LEGACY_NOTIFY_INFO]{VIRTIO_ADMIN_CMD_LEGACY_NOTIFY_INFO} & Query the notification region information \\
\hline
0x0007 & \hyperref[par:Basic Facilities of a Virtio Device / Device groups / Group administration commands / Device and driver capabilities / VIRTIO_ADMIN_CMD_CAP_ID_LIST_QUERY]{VIRTIO_ADMIN_CMD_CAP_ID_LIST_QUERY} & Query the supported device capabilities bitmap \\
\hline
0x0008 & \hyperref[par:Basic Facilities of a Virtio Device / Device groups / Group administration commands / Device and driver capabilities / VIRTIO_ADMIN_CMD_DEVICE_CAP_GET]{VIRTIO_ADMIN_CMD_DEVICE_CAP_GET} & Get the device capabilities \\
\hline
0x0009 & \hyperref[par:Basic Facilities of a Virtio Device / Device groups / Group administration commands / Device and driver capabilities / VIRTIO_ADMIN_CMD_DRIVER_CAP_SET]{VIRTIO_ADMIN_CMD_DRIVER_CAP_SET} & Set the driver capabilities \\
\hline
0x000a & \hyperref[par:Basic Facilities of a Virtio Device / Device groups / Group administration commands / Device resource objects / VIRTIO_ADMIN_CMD_RESOURCE_OBJ_CREATE]{VIRTIO_ADMIN_CMD_RESOURCE_OBJ_CREATE} & Create a device resource object \\
\hline
0x000c & \hyperref[par:Basic Facilities of a Virtio Device / Device groups / Group administration commands / Device resource objects / VIRTIO_ADMIN_CMD_RESOURCE_OBJ_MODIFY]{VIRTIO_ADMIN_CMD_RESOURCE_OBJ_MODIFY} & Modify a device resource object \\
\hline
0x000b & \hyperref[par:Basic Facilities of a Virtio Device / Device groups / Group administration commands / Device resource objects / VIRTIO_ADMIN_CMD_RESOURCE_OBJ_QUERY]{VIRTIO_ADMIN_CMD_RESOURCE_OBJ_QUERY} & Query a device resource object \\
\hline
0x000d & \hyperref[par:Basic Facilities of a Virtio Device / Device groups / Group administration commands / Device resource objects / VIRTIO_ADMIN_CMD_RESOURCE_OBJ_DESTROY]{VIRTIO_ADMIN_CMD_RESOURCE_OBJ_DESTROY} & Destroy a device resource object \\
\hline
0x000e - 0x7FFF & - & Commands using \field{struct virtio_admin_cmd}    \\
\hline
0x8000 - 0xFFFF & - & Reserved for future commands (possibly using a different structure)    \\
\hline
\end{xltabular}

The \field{group_type} specifies the group type identifier.
The \field{group_member_id} specifies the member identifier within the group.
See section \ref{sec:Basic Facilities of a Virtio Device / Device groups}
for the definition of the group type identifier and group member
identifier.

The \field{status} describes the command result and possibly
failure reason at an abstract level, this is appropriate for
forwarding to applications. The \field{status_qualifier} describes
failures at a low virtio specific level, as appropriate for debugging.
The following table describes possible \field{status} values;
to simplify common implementations, they are intentionally
matching common \hyperref[intro:errno]{Linux error names and numbers}:

\begin{tabular}{|l|l|l|}
\hline
Status (decimal) & Name & Description \\
\hline \hline
00   & VIRTIO_ADMIN_STATUS_OK    & successful completion  \\
\hline
06   & VIRTIO_ADMIN_STATUS_ENXIO & no such capability or resource\\
\hline
11   & VIRTIO_ADMIN_STATUS_EAGAIN    & try again \\
\hline
12   & VIRTIO_ADMIN_STATUS_ENOMEM    & insufficient resources \\
\hline
16   & VIRTIO_ADMIN_STATUS_EBUSY     & device busy \\
\hline
22   & VIRTIO_ADMIN_STATUS_EINVAL    & invalid command \\
\hline
28   & VIRTIO_ADMIN_STATUS_ENOSPC    & resources exhausted on device \\
\hline
other   & -    & group administration command error  \\
\hline
\end{tabular}

When \field{status} is VIRTIO_ADMIN_STATUS_OK, \field{status_qualifier}
is reserved and set to zero by the device.

The following table describes possible \field{status_qualifier} values:

\begin{tabularx}{\textwidth}{ |l||l|X| }
\hline
Status & Name & Description \\
\hline \hline
0x00   & VIRTIO_ADMIN_STATUS_Q_OK               & used with VIRTIO_ADMIN_STATUS_OK  \\
\hline
0x01   & VIRTIO_ADMIN_STATUS_Q_INVALID_COMMAND  & command error: no additional information  \\
\hline
0x02   & VIRTIO_ADMIN_STATUS_Q_INVALID_OPCODE   & unsupported or invalid \field{opcode}  \\
\hline
0x03   & VIRTIO_ADMIN_STATUS_Q_INVALID_FIELD    & unsupported or invalid field within \field{command_specific_data}  \\
\hline
0x04   & VIRTIO_ADMIN_STATUS_Q_INVALID_GROUP    & unsupported or invalid \field{group_type} \\
\hline
0x05   & VIRTIO_ADMIN_STATUS_Q_INVALID_MEMBER   & unsupported or invalid \field{group_member_id} \\
\hline
0x06   & VIRTIO_ADMIN_STATUS_Q_NORESOURCE       & out of internal resources: ok to retry \\
\hline
0x07   & VIRTIO_ADMIN_STATUS_Q_TRYAGAIN         & command blocks for too long: should retry \\
\hline
0x08-0xFFFF   & -    & reserved for future use \\
\hline
\end{tabularx}

Each command uses a different \field{command_specific_data} and
\field{command_specific_result} structures and the length of
\field{command_specific_data} and \field{command_specific_result}
depends on these structures and is described separately or is
implicit in the structure description.

Before sending any group administration commands to the device, the driver
needs to communicate to the device which commands it is going to
use. Initially (after reset), only two commands are assumed to be used:
VIRTIO_ADMIN_CMD_LIST_QUERY and VIRTIO_ADMIN_CMD_LIST_USE.

Before sending any other commands for any member of a specific group to
the device, the driver queries the supported commands via
VIRTIO_ADMIN_CMD_LIST_QUERY and sends the commands it is
capable of using via VIRTIO_ADMIN_CMD_LIST_USE.

Commands VIRTIO_ADMIN_CMD_LIST_QUERY and
VIRTIO_ADMIN_CMD_LIST_USE
both use the following structure describing the
command opcodes:

\begin{lstlisting}
struct virtio_admin_cmd_list {
       /* Indicates which of the below fields were returned
       le64 device_admin_cmd_opcodes[];
};
\end{lstlisting}

This structure is an array of 64 bit values in little-endian byte
order, in which a bit is set if the specific command opcode
is supported. Thus, \field{device_admin_cmd_opcodes[0]} refers to the
first 64-bit value in this array corresponding to opcodes 0 to
63, \field{device_admin_cmd_opcodes[1]} is the second 64-bit value
corresponding to opcodes 64 to 127, etc.
For example, the array of size 2 including
the values 0x3 in \field{device_admin_cmd_opcodes[0]}
and 0x1 in \field{device_admin_cmd_opcodes[1]} indicates that only
opcodes 0, 1 and 64 are supported.
The length of the array depends on the supported opcodes - it is
large enough to include bits set for all supported opcodes,
that is the length can be calculated by starting with the largest
supported opcode adding one, dividing by 64 and rounding up.
In other words, for
VIRTIO_ADMIN_CMD_LIST_QUERY and VIRTIO_ADMIN_CMD_LIST_USE the
length of \field{command_specific_result} and
\field{command_specific_data} respectively will be
$DIV_ROUND_UP(max_cmd, 64) * 8$ where DIV_ROUND_UP is integer division
with round up and \field{max_cmd} is the largest available command opcode.

The array is also allowed to be larger and to additionally
include an arbitrary number of all-zero entries.

Accordingly, bits 0 and 1 corresponding to opcode 0
(VIRTIO_ADMIN_CMD_LIST_QUERY) and 1
(VIRTIO_ADMIN_CMD_LIST_USE) are
always set in \field{device_admin_cmd_opcodes[0]} returned by VIRTIO_ADMIN_CMD_LIST_QUERY.

For the command VIRTIO_ADMIN_CMD_LIST_QUERY, \field{opcode} is set to 0x0.
The \field{group_member_id} is unused. It is set to zero by driver.
This command has no command specific data.
The device, upon success, returns a result in
\field{command_specific_result} in the format
\field{struct virtio_admin_cmd_list} describing the
list of group administration commands supported for the group type
specified by \field{group_type}.

For the command VIRTIO_ADMIN_CMD_LIST_USE, \field{opcode}
is set to 0x1.
The \field{group_member_id} is unused. It is set to zero by driver.
The \field{command_specific_data} is in the format
\field{struct virtio_admin_cmd_list} describing the
list of group administration commands used by the driver
with the group type specified by \field{group_type}.

This command has no command specific result.

The driver issues the command VIRTIO_ADMIN_CMD_LIST_QUERY to
query the list of commands valid for this group and before sending
any commands for any member of a group.

The driver then enables use of some of the opcodes by sending to
the device the command VIRTIO_ADMIN_CMD_LIST_USE with a subset
of the list returned by VIRTIO_ADMIN_CMD_LIST_QUERY that is
both understood and used by the driver.

If the device supports the command list used by the driver, the
device completes the command with status VIRTIO_ADMIN_STATUS_OK.
If the device does not support the command list
(for example, if the driver is not capable to use
some required commands), the device
completes the command with status
VIRTIO_ADMIN_STATUS_INVALID_FIELD.

Note: the driver is assumed not to set bits in
device_admin_cmd_opcodes
if it is not familiar with how the command opcode
is used, since the device could have dependencies between
command opcodes.

It is assumed that all members in a group support and are used
with the same list of commands. However, for owner devices
supporting multiple group types, the list of supported commands
might differ between different group types.

\subsubsection{Legacy Interfaces}\label{sec:Basic Facilities of a Virtio Device / Device groups / Group
administration commands / Legacy Interface}

In some systems, there is a need to support utilizing a legacy driver with
a device that does not directly support the legacy interface. In such scenarios,
a group owner device can provide the legacy interface functionality for the
group member devices. The driver of the owner device can then access the legacy
interface of a member device on behalf of the legacy member device driver.

For example, with the SR-IOV group type, group members (VFs) can not present
the legacy interface in an I/O BAR in BAR0 as expected by the legacy pci driver.
If the legacy driver is running inside a virtual machine, the hypervisor
executing the virtual machine can present a virtual device with an I/O BAR in
BAR0. The hypervisor intercepts the legacy driver accesses to this I/O BAR and
forwards them to the group owner device (PF) using group administration commands.

The following commands support such a legacy interface functionality:

\begin{enumerate}
\item VIRTIO_ADMIN_CMD_LEGACY_COMMON_CFG_WRITE
\item VIRTIO_ADMIN_CMD_LEGACY_COMMON_CFG_READ
\item VIRTIO_ADMIN_CMD_LEGACY_DEV_CFG_WRITE
\item VIRTIO_ADMIN_CMD_LEGACY_DEV_CFG_READ
\end{enumerate}

These commands are currently only defined for the SR-IOV group type and
have, generally, the same effect as member device accesses through a legacy
interface listed in section \ref{sec:Virtio Transport Options / Virtio Over PCI
Bus / PCI Device Layout / Legacy Interfaces: A Note on PCI Device Layout} except
that little-endian format is assumed unconditionally.

\paragraph{VIRTIO_ADMIN_CMD_LEGACY_COMMON_CFG_WRITE}
\label{par:Basic Facilities of a Virtio Device / Device groups / Group administration commands / Legacy Interface / VIRTIO_ADMIN_CMD_LEGACY_COMMON_CFG_WRITE}

This command has the same effect as writing into the virtio common configuration
structure through the legacy interface. The \field{command_specific_data} is in
the format \field{struct virtio_admin_cmd_legacy_common_cfg_wr_data} describing
the access to be performed.

\begin{lstlisting}
struct virtio_admin_cmd_legacy_common_cfg_wr_data {
        u8 offset; /* Starting byte offset within the common configuration structure to write */
        u8 reserved[7];
        u8 data[];
};
\end{lstlisting}

For the command VIRTIO_ADMIN_CMD_LEGACY_COMMON_CFG_WRITE, \field{opcode}
is set to 0x2.
The \field{group_member_id} refers to the member device to be accessed.
The \field{offset} refers to the offset for the write within the virtio common
configuration structure, and excluding the device-specific configuration.
The length of the data to write is simply the length of \field{data}.

No length or alignment restrictions are placed on the value of the
\field{offset} and the length of the \field{data}, except that the resulting
access refers to a single field and is completely within the virtio common
configuration structure, excluding the device-specific configuration.

This command has no command specific result.

\paragraph{VIRTIO_ADMIN_CMD_LEGACY_COMMON_CFG_READ}
\label{par:Basic Facilities of a Virtio Device / Device groups / Group administration commands / Legacy Interface / VIRTIO_ADMIN_CMD_LEGACY_COMMON_CFG_READ}

This command has the same effect as reading from the virtio common configuration
structure through the legacy interface. The \field{command_specific_data} is in
the format \field{struct virtio_admin_cmd_legacy_common_cfg_rd_data} describing
the access to be performed.

\begin{lstlisting}
struct virtio_admin_cmd_legacy_common_cfg_rd_data {
        u8 offset; /* Starting byte offset within the common configuration structure to read */
};
\end{lstlisting}

For the command VIRTIO_ADMIN_CMD_LEGACY_COMMON_CFG_READ, \field{opcode}
is set to 0x3.
The \field{group_member_id} refers to the member device to be accessed.
The \field{offset} refers to the offset for the read from the virtio common
configuration structure, and excluding the device-specific configuration.

\begin{lstlisting}
struct virtio_admin_cmd_legacy_common_cfg_rd_result {
        u8 data[];
};
\end{lstlisting}

No length or alignment restrictions are placed on the value of the
\field{offset} and the length of the \field{data}, except that the resulting
access refers to a single field and is completely within the virtio common
configuration structure, excluding the device-specific configuration.

When the command completes successfully, \field{command_specific_result}
is in the format \field{struct virtio_admin_cmd_legacy_common_cfg_rd_result}
returned by the device. The length of the data read is simply the length of
\field{data}.

\paragraph{VIRTIO_ADMIN_CMD_LEGACY_DEV_CFG_WRITE}
\label{par:Basic Facilities of a Virtio Device / Device groups / Group administration commands / Legacy Interface / VIRTIO_ADMIN_CMD_LEGACY_DEV_CFG_WRITE}

This command has the same effect as writing into the virtio device-specific
configuration through the legacy interface. The \field{command_specific_data} is in
the format \field{struct virtio_admin_cmd_legacy_dev_reg_wr_data} describing
the access to be performed.

\begin{lstlisting}
struct virtio_admin_cmd_legacy_dev_reg_wr_data {
        u8 offset; /* Starting byte offset within the device-specific configuration to write */
        u8 reserved[7];
        u8 data[];
};
\end{lstlisting}

For the command VIRTIO_ADMIN_CMD_LEGACY_DEV_CFG_WRITE, \field{opcode}
is set to 0x4.
The \field{group_member_id} refers to the member device to be accessed.
The \field{offset} refers to the offset for the write within the virtio
device-specific configuration. The length of the data to write is simply
the length of \field{data}.

No length or alignment restrictions are placed on the value of the
\field{offset} and the length of the \field{data}, except that the resulting
access refers to a single field and is completely within the device-specific
configuration.

This command has no command specific result.

\paragraph{VIRTIO_ADMIN_CMD_LEGACY_DEV_CFG_READ}
\label{par:Basic Facilities of a Virtio Device / Device groups / Group administration commands / Legacy Interface / VIRTIO_ADMIN_CMD_LEGACY_DEV_CFG_READ}

This command has the same effect as reading from the virtio device-specific
configuration through the legacy interface. The \field{command_specific_data} is in
the format \field{struct virtio_admin_cmd_legacy_common_cfg_rd_data} describing
the access to be performed.

\begin{lstlisting}
struct virtio_admin_cmd_legacy_dev_cfg_rd_data {
        u8 offset; /* Starting byte offset within the device-specific configuration to read */
};
\end{lstlisting}

For the command VIRTIO_ADMIN_CMD_LEGACY_DEV_CFG_READ, \field{opcode}
is set to 0x5.
The \field{group_member_id} refers to the member device to be accessed.
The \field{offset} refers to the offset for the read from the virtio device-specific
configuration.

\begin{lstlisting}
struct virtio_admin_cmd_legacy_dev_reg_rd_result {
        u8 data[];
};
\end{lstlisting}

No length or alignment restrictions are placed on the value of the
\field{offset} and the length of the \field{data}, except that the resulting
access refers to a single field and is completely within the device-specific
configuration.

When the command completes successfully, \field{command_specific_result} is in
the format \field{struct virtio_admin_cmd_legacy_dev_reg_rd_result}
returned by the device.

The length of the data read is simply the length of \field{data}.

\paragraph{VIRTIO_ADMIN_CMD_LEGACY_NOTIFY_INFO}
\label{par:Basic Facilities of a Virtio Device / Device groups / Group administration commands / Legacy Interface / VIRTIO_ADMIN_CMD_LEGACY_NOTIFY_INFO}

The driver of the owner device can send a driver notification to the member
device operated using the legacy interface by executing
VIRTIO_ADMIN_CMD_LEGACY_COMMON_CFG_WRITE with the \field{offset} matching
\field{Queue Notify} and the \field{data} containing a 16-bit virtqueue index to
be notified.

However, as VIRTIO_ADMIN_CMD_LEGACY_COMMON_CFG_WRITE is also used for slow path
configuration a separate dedicated mechanism for sending such driver
notifications to the member device can be made available by the owner device.
For the SR-IOV group type, the optional command
VIRTIO_ADMIN_CMD_LEGACY_NOTIFY_INFO addresses this need by returning to the
driver one or more addresses which can be used to send such driver
notifications. The notification address returned can be in the device memory
(PCI BAR or VF BAR) of the device.

In this alternative approach, driver notifications are sent by
writing a 16-bit virtqueue index to be notified, in the little-endian
format, to the notification address returned by
the VIRTIO_ADMIN_CMD_LEGACY_NOTIFY_INFO command.

Any driver notification sent through the notification address has the same effect
as if it was sent using the VIRTIO_ADMIN_CMD_LEGACY_COMMON_CFG_WRITE command with
the \field{offset} matching \field{Queue Notify}.

This command is only defined for the SR-IOV group type.

For the command VIRTIO_ADMIN_CMD_LEGACY_NOTIFY_INFO, \field{opcode}
is set to 0x6.
The \field{group_member_id} refers to the member device to be accessed.
This command does not use \field{command_specific_data}.

When the device supports the VIRTIO_ADMIN_CMD_LEGACY_NOTIFY_INFO command, the
group owner device hardwires VF BAR0 to zero in the SR-IOV Extended capability.

\begin{lstlisting}
struct virtio_pci_legacy_notify_info {
        u8 flags;  /* 0 = end of list, 1 = owner device, 2 = member device */
        u8 bar;    /* BAR of the member or the owner device */
        u8 padding[6];
        le64 offset; /* Offset within bar. */
};

struct virtio_admin_cmd_legacy_notify_info_result {
        struct virtio_pci_legacy_notify_info entries[4];
};
\end{lstlisting}

A \field{flags} value of 0x1 indicates that the notification address is of
the owner device, the value of 0x2 indicates that the notification address is of
the member device and the value of 0x0 indicates that all the entries starting
from that entry are invalid entries in \field{entries}. All other values in
\field{flags} are reserved.

The \field{bar} values 0x1 to 0x5 specify BAR1 to BAR5 respectively:
when the \field{flags} is 0x1 this is specified by the Base Address Registers
in the PCI header of the device,
when the \field{flags} is 0x2 this is specified by the VF BARn
registers in the SR-IOV Extended Capability of the device.

The \field{offset} indicates the notification address relative to BAR indicated
in \field{bar}. This value is 2-byte aligned.

When the command completes successfully, \field{command_specific_result} is in
the format \field{struct virtio_admin_cmd_legacy_notify_info_result}. The
device can supply up to 4 entries each with a different notification
address. In this case, any of the entries can be used by the driver. The order
of the entries serves as a preference hint to the driver. The driver is expected
to utilize the entries placed earlier in the array in preference to the later
ones. The driver is also expected to ignore any invalid entries, as well as
the end of list entry if present and any entries following the end of list.

\devicenormative{\paragraph}{Legacy Interface}{Basic Facilities of a Virtio Device / Device groups / Group administration commands / Legacy Interface}

A device MUST either support all of, or none of
VIRTIO_ADMIN_CMD_LEGACY_COMMON_CFG_WRITE,
VIRTIO_ADMIN_CMD_LEGACY_COMMON_CFG_READ,
VIRTIO_ADMIN_CMD_LEGACY_DEV_CFG_WRITE and
VIRTIO_ADMIN_CMD_LEGACY_DEV_CFG_READ commands.

For VIRTIO_ADMIN_CMD_LEGACY_COMMON_CFG_WRITE,
VIRTIO_ADMIN_CMD_LEGACY_COMMON_CFG_READ,
VIRTIO_ADMIN_CMD_LEGACY_DEV_CFG_WRITE and
VIRTIO_ADMIN_CMD_LEGACY_DEV_CFG_READ commands,
the device MUST decode and encode (respectively) the value of the
\field{data} using the little-endian format.

For the VIRTIO_ADMIN_CMD_LEGACY_COMMON_CFG_WRITE and
VIRTIO_ADMIN_CMD_LEGACY_COMMON_CFG_READ commands, 
the device MUST fail the command when the value of the
\field{offset} and the length of the \field{data} do not refer to a
single field or are not completely within the virtio common configuration
excluding the device-specific configuration.

For the VIRTIO_ADMIN_CMD_LEGACY_DEV_CFG_WRITE and
VIRTIO_ADMIN_CMD_LEGACY_DEV_CFG_READ commands,
the device MUST fail the command when the value of the
\field{offset} and the length of the \field{data} do not refer to a
single field or are not completely within the virtio device-specific
configuration.

The command VIRTIO_ADMIN_CMD_LEGACY_COMMON_CFG_WRITE MUST have the same effect
as writing into the virtio common configuration structure through the legacy
interface.

The command VIRTIO_ADMIN_CMD_LEGACY_COMMON_CFG_READ MUST have the same effect as
reading from the virtio common configuration structure through the legacy
interface.

The command VIRTIO_ADMIN_CMD_LEGACY_DEV_CFG_WRITE MUST have the same effect as
writing into the virtio device-specific configuration through the legacy
interface.

The command VIRTIO_ADMIN_CMD_LEGACY_DEV_CFG_READ MUST have the same effect as
reading from the virtio device-specific configuration through the legacy
interface.

For the SR-IOV group type, when the owner device supports
VIRTIO_ADMIN_CMD_LEGACY_COMMON_CFG_READ,
VIRTIO_ADMIN_CMD_LEGACY_COMMON_CFG_WRITE, VIRTIO_ADMIN_CMD_LEGACY_DEV_CFG_READ,
VIRTIO_ADMIN_CMD_LEGACY_DEV_CFG_WRITE and VIRTIO_ADMIN_CMD_LEGACY_NOTIFY_INFO
commands,
\begin{itemize}
\item the owner device and the group member device SHOULD follow the rules
for the PCI Revision ID and Subsystem Device ID of the non-transitional devices
documented in section \ref{sec:Virtio Transport Options / Virtio Over PCI Bus / PCI Device Discovery}.

\item the owner device SHOULD follow the rules for the PCI Device ID of the non-transitional
devices documented in section
\ref{sec:Virtio Transport Options / Virtio Over PCI Bus / PCI Device Discovery}.

\item any driver notification received by the device at any of the notification
address supplied in the command result of
VIRTIO_ADMIN_CMD_LEGACY_NOTIFY_INFO MUST function as if the device received
the notification through VIRTIO_ADMIN_CMD_LEGACY_COMMON_CFG_WRITE
command at an offset \field{offset} matching \field{Queue Notify}.
\end{itemize}

If the device supports the VIRTIO_ADMIN_CMD_LEGACY_NOTIFY_INFO command,
\begin{itemize}
\item the device MUST also support all of VIRTIO_ADMIN_CMD_LEGACY_COMMON_CFG_WRITE,
VIRTIO_ADMIN_CMD_LEGACY_COMMON_CFG_READ,
VIRTIO_ADMIN_CMD_LEGACY_DEV_CFG_WRITE and
VIRTIO_ADMIN_CMD_LEGACY_DEV_CFG_READ commands.

\item in the command result of VIRTIO_ADMIN_CMD_LEGACY_NOTIFY_INFO, the last
\field{struct virtio_pci_legacy_notify_info} entry MUST have \field{flags} of
zero.

\item in the command result of VIRTIO_ADMIN_CMD_LEGACY_NOTIFY_INFO, valid
entries MUST have a \field{bar} which is not hardwired to zero.

\item in the command result of VIRTIO_ADMIN_CMD_LEGACY_NOTIFY_INFO, valid
entries MUST have an \field{offset} aligned to 2-byte.

\item the device MAY support VIRTIO_ADMIN_CMD_LEGACY_NOTIFY_INFO with entries
of the owner device or the member device or both of them.

\item for the SR-IOV group type, the group owner device MUST hardwire VF BAR0
to zero in the SR-IOV Extended capability.
\end{itemize}

\drivernormative{\paragraph}{Legacy Interface}{Basic Facilities of a Virtio Device / Device groups / Group administration commands / Legacy Interface}

For VIRTIO_ADMIN_CMD_LEGACY_COMMON_CFG_WRITE,
VIRTIO_ADMIN_CMD_LEGACY_COMMON_CFG_READ,
VIRTIO_ADMIN_CMD_LEGACY_DEV_CFG_WRITE and
VIRTIO_ADMIN_CMD_LEGACY_DEV_CFG_READ commands,
the driver MUST encode and decode (respectively) the value of the \field{data}
using the little-endian format.

For the VIRTIO_ADMIN_CMD_LEGACY_COMMON_CFG_WRITE and
VIRTIO_ADMIN_CMD_LEGACY_COMMON_CFG_READ commands,
the driver SHOULD set \field{offset} and the length of the \field{data}
to refer to a single field within the virtio common configuration structure
excluding the device-specific configuration.

For the VIRTIO_ADMIN_CMD_LEGACY_DEV_CFG_WRITE and
VIRTIO_ADMIN_CMD_LEGACY_DEV_CFG_READ commands,
the driver SHOULD set \field{offset} and the length of the \field{data}
to refer to a single field within device specific configuration.

If VIRTIO_ADMIN_CMD_LEGACY_NOTIFY_INFO command is supported, the driver
SHOULD use the notification address to send all driver notifications to the
device.

If within \field{struct virtio_admin_cmd_legacy_notify_info_result} returned by
VIRTIO_ADMIN_CMD_LEGACY_NOTIFY_INFO, the \field{flags} value
for a specific \field{struct virtio_pci_legacy_notify_info} entry is 0x0, the
driver MUST ignore this entry and all the following \field{entries}.
Additionally, for all other entries, the driver MUST validate that
\begin{itemize}
\item the \field{flags} is either 0x1 or 0x2
\item the \field{bar} corresponds to a valid BAR of either the owner or the
member device, depending on the \field{flags}
\item the \field{offset} is 2-byte aligned and corresponds to an address
within the BAR specified by the \field{bar}
on \field{flags}
\end{itemize}, any entry which does not meet these constraints MUST be ignored
by the driver.

\subsubsection{Device and driver capabilities}\label{sec:Basic Facilities of a Virtio Device / Device groups / Group administration commands / Device and driver capabilities}

Device and driver capabilities are implemented as structured groupings for
specific device functionality and their related resource objects. The device exposes
its supported functionality and resource object limits through an administration
command, utilizing the 'self group type.' Each capability possesses a
unique ID. Through an administration command, also employing the
'self group type,' the driver reports the functionality and
resource object limits it intends to use. Before executing any operations
related to the capabilities, the driver communicates these
capabilities to the device. The driver is allowed to set the
capability at any time, provided there are no pending operations
at the device level associated with that capability.

The device presents the supported capability IDs to the driver as a bitmap.
The driver uses the administration command to learn about the
supported capabilities bitmap.

A capability consists of one or more fields, where each field can be a
limit number, a bitmap, or an array of entries. In an array field,
the structure depends on the specific array and the capability type.
For each bitmap field, the driver sets the desired bits - but only out of
those bits in a bitmap that the device has presented.
The driver sets each limit number field to a desired value that
is smaller than or equal to the value the device presented.
Similarly, for an array field, the driver sets the desired capability
entries but only out of the capability entries that the device has presented.

It is anticipated that any necessary new fields for a capability will be
appended to the structure's end, ensuring both forward and backward
compatibility between the device and driver. Furthermore, to avoid
indefinite growth of a single capability, it is expected that new
functionality will lead to the creation of new capability rather
than expanding existing ones.

Capabilities are categorized into two ranges by their IDs, as listed:

\begin{table}[H]
\caption{Capability ids}
\label{table:Basic Facilities of a Virtio Device / Device groups / Group administration commands / Device and driver capabilities / capability ids}
\begin{tabularx}{\textwidth}{ |l|X| }
\hline
Id & Description  \\
\hline \hline
0x0000-0x07ff & Generic capability for all device types \\
\hline
0x0800-0x0fff & Device type specific capability \\
\hline
0x1000 - 0xFFFF & Reserved for future \\
\hline
\end{tabularx}
\end{table}

Common capabilities are listed:

\begin{table}[H]
\caption{Common capability ids}
\label{table:Basic Facilities of a Virtio Device / Device groups / Group administration commands / Device and driver capabilities / Common capability ids}
\begin{tabularx}{\textwidth}{ |l|l|X| }
\hline
Id & Name & Description  \\
\hline \hline
0x0000 & \hyperref[par:Basic Facilities of a Virtio Device / Device groups / Group administration commands / Device parts / VIRTIO_DEV_PARTS_CAP]{VIRTIO_DEV_PARTS_CAP} & Device parts capability \\
\hline
0x0001-0x07ff & - & Generic capability for all device types \\
\hline
\end{tabularx}
\end{table}

Device type specific capabilities are described separately for each device
type under \field{Device and driver capabilities}.

The device and driver capabilities commands are currently defined for self group
type.

\begin{enumerate}
\item VIRTIO_ADMIN_CMD_CAP_ID_LIST_QUERY
\item VIRTIO_ADMIN_CMD_DEVICE_CAP_GET
\item VIRTIO_ADMIN_CMD_DRIVER_CAP_SET
\end{enumerate}

\paragraph{VIRTIO_ADMIN_CMD_CAP_ID_LIST_QUERY}\label{par:Basic Facilities of a Virtio Device / Device groups / Group administration commands / Device and driver capabilities / VIRTIO_ADMIN_CMD_CAP_ID_LIST_QUERY}

This command queries the bitmap of capability ids
listed in \ref{table:Basic Facilities of a Virtio Device / Device groups / Group administration commands / Device and driver capabilities / capability ids}.

For the command VIRTIO_ADMIN_CMD_CAP_ID_LIST_QUERY, \field{opcode} is set to 0x7.
\field{group_member_id} is set to zero.

This command has no command specific data.

\begin{lstlisting}
struct virtio_admin_cmd_query_cap_id_result {
        le64 supported_caps[];
};
\end{lstlisting}

When the command completes successfully, \field{command_specific_result}
is in the format \field{struct virtio_admin_cmd_query_cap_id_result}.

\field{supported_caps} is an array of 64 bit values in little-endian byte
order, in which a bit is set if the specific capability is supported.
Thus, \field{supported_caps[0]} refers to the first 64-bit value in this
array corresponding to capability ids 0 to
63, \field{supported_caps[1]} is the second 64-bit value corresponding to
capability ids 64 to 127, etc. For example, the array of size 2 including
the values 0x3 in \field{supported_caps[0]} and 0x1 in
\field{supported_caps[1]} indicates that only capability id 0, 1
and 64 are supported.
The length of the array depends on the supported capabilities - it is
large enough to include bits set for all supported capability ids,
that is the length can be calculated by starting with the largest
supported capability id adding one, dividing by 64 and rounding up.
In other words, for VIRTIO_ADMIN_CMD_CAP_ID_LIST_QUERY the length of
\field{command_specific_result} will be
$DIV_ROUND_UP(max_cap_id, 64) * 8$ where DIV_ROUND_UP is integer division
with round up and \field{max_cap_id} is the largest available capability id.

The array is also allowed to be larger and to additionally include an arbitrary
number of all-zero entries.

\paragraph{VIRTIO_ADMIN_CMD_DEVICE_CAP_GET}\label{par:Basic Facilities of a Virtio Device / Device groups / Group administration commands / Device and driver capabilities / VIRTIO_ADMIN_CMD_DEVICE_CAP_GET}

This command gets the device capability for the specified capability
id \field{id}.

For the command VIRTIO_ADMIN_CMD_DEVICE_CAP_GET, \field{opcode} is set to 0x8.
\field{group_member_id} is set to zero.

\field{command_specific_data} is in format
\field{struct virtio_admin_cmd_cap_get_data}.

\begin{lstlisting}
struct virtio_admin_cmd_cap_get_data {
        le16 id;
        u8 reserved[6];
};
\end{lstlisting}

\field{id} refers to the capability id listed in \ref{table:Basic Facilities of a Virtio Device / Device groups / Group administration commands / Device and driver capabilities / capability ids}.
\field{reserved} is reserved for future use and set to zero.

\begin{lstlisting}
struct virtio_admin_cmd_cap_get_result {
        u8 cap_specific_data[];
};
\end{lstlisting}

When the command completes successfully, \field{command_specific_result}
is in the format \field{struct virtio_admin_cmd_cap_get_result} responded
by the device. Each capability uses different capability specific
\field{cap_specific_data} and is described separately.

\paragraph{VIRTIO_ADMIN_CMD_DRIVER_CAP_SET}\label{par:Basic Facilities of a Virtio Device / Device groups / Group administration commands / Device and driver capabilities / VIRTIO_ADMIN_CMD_DRIVER_CAP_SET}

This command sets the driver capability, indicating to the device which capability
the driver uses. The driver can set a resource object limit capability that is smaller
than or equal to the value published by the device capability.
If the capability is a set of flags, the driver sets the flag bits that are set
in the device capability; the driver does not set any flag bits that are not
set by the device.

For the command VIRTIO_ADMIN_CMD_DRIVER_CAP_SET, \field{opcode} is set to 0x9.
\field{group_member_id} is set to zero.
The \field{command_specific_data} is in the format
\field{struct virtio_admin_cmd_cap_set_data}.

\begin{lstlisting}
struct virtio_admin_cmd_cap_set_data {
        le16 id;
        u8 reserved[6];
        u8 cap_specific_data[];
};
\end{lstlisting}

\field{id} refers to the capability id listed in \ref{table:Basic Facilities of a Virtio Device / Device groups / Group administration commands / Device and driver capabilities / capability ids}.
\field{reserved} is reserved for future use and set to zero.

There is no command specific result.
When the command completes successfully, the driver capability is updated to
the values supplied in \field{cap_specific_data}.

\devicenormative{\paragraph}{Device and driver capabilities}{Basic Facilities of a Virtio Device / Device groups / Group administration commands / Device and driver capabilities}

If the device supports capabilities, it MUST support the commands
VIRTIO_ADMIN_CMD_CAP_ID_LIST_QUERY,
VIRTIO_ADMIN_CMD_DRIVER_CAP_SET, and
VIRTIO_ADMIN_CMD_DEVICE_CAP_GET.

For the VIRTIO_ADMIN_CMD_DRIVER_CAP_SET command,
\begin{itemize}
\item the device MUST support the setting of resource object limit driver capability to a
value that is same as or smaller than the one reported in the device
capability,
\item the device MUST support the setting of capability flags bits to
all or fewer bits than the one reported in the device capability;
\end{itemize}
this is applicable unless specific capability fields are explicitly
stated as non-writable in the VIRTIO_ADMIN_CMD_DEVICE_CAP_GET command.

The device MAY complete the command VIRTIO_ADMIN_CMD_DRIVER_CAP_SET with
\field{status} set to VIRTIO_ADMIN_STATUS_EINVAL,
if the capability resource object limit is larger than the value reported by the
device's capability, or the capability flag bit is set, which is not set in
the device's capability.

The device MUST complete the commands VIRTIO_ADMIN_CMD_CAP_ID_LIST_QUERY,
VIRTIO_ADMIN_CMD_DRIVER_CAP_GET, and VIRTIO_ADMIN_CMD_DRIVER_CAP_SET
with \field{status} set to VIRTIO_ADMIN_STATUS_EINVAL if the commands are not
for the self group type.

The device SHOULD complete the commands VIRTIO_ADMIN_CMD_CAP_ID_LIST_QUERY,
VIRTIO_ADMIN_CMD_DRIVER_CAP_GET, VIRTIO_ADMIN_CMD_DRIVER_CAP_SET with
\field{status} set to VIRTIO_ADMIN_STATUS_EINVAL if the commands are not for the
self group type.

The device SHOULD complete the command VIRTIO_ADMIN_CMD_DRIVER_CAP_SET with
\field{status} set to VIRTIO_ADMIN_STATUS_EBUSY if the command requests to disable
a capability while the device still has valid resource objects related to the
capability being disabled.

The device SHOULD complete the comands VIRTIO_ADMIN_CMD_DEVICE_CAP_GET and
VIRTIO_ADMIN_CMD_DRIVER_CAP_SET with \field{status} set to
VIRTIO_ADMIN_STATUS_ENXIO if the capability id is not reported
in command VIRTIO_ADMIN_CMD_CAP_ID_LIST_QUERY.

Upon a device reset, the device MUST reset all driver capabilities.

The device SHOULD treat the driver resource limits as zero if the
driver has not set such capability, unless otherwise explicitly stated.

\drivernormative{\paragraph}{Device and driver capabilities}{Basic Facilities of a Virtio Device / Device groups / Group administration commands / Device and driver capabilities}

The driver MUST send the command VIRTIO_ADMIN_CMD_DRIVER_CAP_SET before
using any resource objects that depend on such a capability.

In VIRTIO_ADMIN_CMD_DRIVER_CAP_SET command, the driver MUST NOT set
\begin{itemize}
\item the resource object limit value larger than the value reported
by the device in the command VIRTIO_ADMIN_CMD_DEVICE_CAP_GET,
\item flags bits which was not reported by the device in the command
VIRTIO_ADMIN_CMD_DEVICE_CAP_GET,
\item array entries not reported by the device in the command
VIRTIO_ADMIN_CMD_DEVICE_CAP_GET.
\end{itemize}

The driver MUST NOT disable any of the driver capability using the command
VIRTIO_ADMIN_CMD_DRIVER_CAP_SET when related resource objects
are created but not destroyed.

\subsubsection{Device resource objects}\label{sec:Basic Facilities of a Virtio Device / Device groups / Group administration commands / Device resource objects}

Providing certain functionality consumes limited device resources such as
memory, processing units, buffer memory, or end-to-end credits. A device may
support multiple types of resource objects, each controlling different device
functionality. To manage this, virtio provides
\field{Device resource objects} that the driver can create, modify, and
destroy using administration commands with the self group type. Creating and
destroying a resource object consume and release device resources, respectively.
The device resource object query command returns the resource object as
maintained by the device.

For each resource type, the number of resource objects that can be created
is reported by the device as part of a device capability
\ref{sec:Basic Facilities of a Virtio Device / Device groups / Group administration commands / Device and driver capabilities}.
The driver reports the desired (same or lower) number of resource objects
as part of a driver capability \ref{sec:Basic Facilities of a Virtio Device / Device groups / Group administration commands / Device and driver capabilities}.
For each device object type, resource object limit is defined by field
\field{limit} using \field{Device and driver capabilities}.

\begin{lstlisting}
le32 limit; /* maximum resource id = limit - 1 */
\end{lstlisting}

Each resource object has a unique resource object ID - a driver-assigned number
in the range of 0 to \field{limit - 1}, where the \field{limit} is the maximum
number set by the driver for this resource object type. These resource IDs are unique within
each resource object type. The driver assigns the resource ID when creating a
device resource object. Once the resource object is successfully created,
subsequent resource modification, query, and destroy commands use this
resource object ID. No two resource objects share the same ID. Destroying a
resource object allows for the reuse of its ID for another resource object
of the same type.

A valid resource object id is \field{limit - 1}. For example, when a device
reports a \field{limit = 10} capability for a resource object, and drivers sets
\field{limit = 8}, the valid resource object id range for the device and the
driver is 0 to 7 for all the resource object commands. In this example,
the driver can only create 8 resource objects of a specified type.

A resource object of one type may depend on the resource object of another type.
Such dependency between resource objects is established by referring to the unique
resource ID in the administration commands. For example, a driver creates a
resource object identified by ID A of one type, then creates another resource
object identified by ID B of a different type, which depends on resource object
A. This dependency establishes the lifecycle of these resource objects. The driver
that creates the dependent resource object must destroy the resource objects in the
exact reverse order of their creation. In this example, the driver would
destroy resource object B before destroying resource object A.

Some resource object types are generic, common across multiple devices.
Others are specific for one device type.

\begin{tabular}{|l|l|}
\hline
Resource object type & Description \\
\hline \hline
0x000-0x1ff & Generic resource object type common across all devices \\
\hline
0x200-0x4ff & Device type specific resource object \\
\hline
0x500-0xffff & Reserved for future use  \\
\hline
\end{tabular}

When the device resets, it starts with zero resources of each type; the driver
can create resources up to the published \field{limit}. The driver can
destroy and recreate the resource one or multiple times. Upon device reset,
all resource objects created by the driver are destroyed within the device.


\devicenormative{\subsubsection}{Group administration commands}{Basic Facilities of a Virtio Device / Device groups / Group administration commands}

The device MUST validate \field{opcode}, \field{group_type} and
\field{group_member_id}, and if any of these has an invalid or
unsupported value, set \field{status} to
VIRTIO_ADMIN_STATUS_EINVAL and set \field{status_qualifier}
accordingly:
\begin{itemize}
\item if \field{group_type} is invalid, \field{status_qualifier}
	MUST be set to VIRTIO_ADMIN_STATUS_Q_INVALID_GROUP;
\item otherwise, if \field{opcode} is invalid,
	\field{status_qualifier} MUST be set to
	VIRTIO_ADMIN_STATUS_Q_INVALID_OPCODE;
\item otherwise, if \field{group_member_id} is used by the
	specific command and is invalid, \field{status_qualifier} MUST be
	set to VIRTIO_ADMIN_STATUS_Q_INVALID_MEMBER.
\end{itemize}

If a command completes successfully, the device MUST set
\field{status} to VIRTIO_ADMIN_STATUS_OK.

If a command fails, the device MUST set
\field{status} to a value different from VIRTIO_ADMIN_STATUS_OK.

If \field{status} is set to VIRTIO_ADMIN_STATUS_EINVAL, the
device state MUST NOT change, that is the command MUST NOT have
any side effects on the device, in particular the device MUST NOT
enter an error state as a result of this command.

If a command fails, the device state generally SHOULD NOT change,
as far as possible.

The device MAY enforce additional restrictions and dependencies on
opcodes used by the driver and MAY fail the command
VIRTIO_ADMIN_CMD_LIST_USE with \field{status} set to VIRTIO_ADMIN_STATUS_EINVAL
and \field{status_qualifier} set to VIRTIO_ADMIN_STATUS_Q_INVALID_FIELD
if the list of commands used violate internal device dependencies.

If the device supports multiple group types, commands for each group
type MUST operate independently of each other, in particular,
the device MAY return different results for VIRTIO_ADMIN_CMD_LIST_QUERY
for different group types.

After reset, if the device supports a given group type
and before receiving VIRTIO_ADMIN_CMD_LIST_USE for this group type
the device MUST assume
that the list of legal commands used by the driver consists of
the two commands VIRTIO_ADMIN_CMD_LIST_QUERY and VIRTIO_ADMIN_CMD_LIST_USE.

After completing VIRTIO_ADMIN_CMD_LIST_USE successfully,
the device MUST set the list of legal commands used by the driver
to the one supplied in \field{command_specific_data}.

The device MUST validate commands against the list used by
the driver and MUST fail any commands not in the list with
\field{status} set to VIRTIO_ADMIN_STATUS_EINVAL
and \field{status_qualifier} set to
VIRTIO_ADMIN_STATUS_Q_INVALID_OPCODE.

The list of supported commands reported by the device MUST NOT
shrink (but MAY expand): after reporting a given command as
supported through VIRTIO_ADMIN_CMD_LIST_QUERY the device MUST NOT
later report it as unsupported.  Further, after a given set of
commands has been used (via a successful
VIRTIO_ADMIN_CMD_LIST_USE), then after a device or system reset
the device SHOULD complete successfully any following calls to
VIRTIO_ADMIN_CMD_LIST_USE with the same list of commands; if this
command VIRTIO_ADMIN_CMD_LIST_USE fails after a device or system
reset, the device MUST not fail it solely because of the command
list used.  Failure to do so would interfere with resuming from
suspend and error recovery. Exceptions MAY apply if the system
configuration assures, in some way, that the driver does not
cache the previous value of VIRTIO_ADMIN_CMD_LIST_USE,
such as in the case of a firmware upgrade or downgrade.

When processing a command with the SR-IOV group type,
if the device does not have an SR-IOV Extended Capability or
if \field{VF Enable} is clear
then the device MUST fail all commands with
\field{status} set to VIRTIO_ADMIN_STATUS_EINVAL and
\field{status_qualifier} set to
VIRTIO_ADMIN_STATUS_Q_INVALID_GROUP;
otherwise, if \field{group_member_id} is not
between $1$ and \field{NumVFs} inclusive,
the device MUST fail all commands with
\field{status} set to VIRTIO_ADMIN_STATUS_EINVAL and
\field{status_qualifier} set to
VIRTIO_ADMIN_STATUS_Q_INVALID_MEMBER;
\field{NumVFs}, \field{VF Migration Capable}  and
\field{VF Enable} refer to registers within the SR-IOV Extended
Capability as specified by \hyperref[intro:PCIe]{[PCIe]}.

\drivernormative{\subsubsection}{Group administration commands}{Basic Facilities of a Virtio Device / Device groups / Group administration commands}

The driver MAY discover whether device supports a specific group type
by issuing VIRTIO_ADMIN_CMD_LIST_QUERY with the matching
\field{group_type}.

The driver MUST issue VIRTIO_ADMIN_CMD_LIST_USE
and wait for it to be completed with status
VIRTIO_ADMIN_STATUS_OK before issuing any commands
(except for the initial VIRTIO_ADMIN_CMD_LIST_QUERY
and VIRTIO_ADMIN_CMD_LIST_USE).

The driver MAY issue VIRTIO_ADMIN_CMD_LIST_USE any number
of times but MUST NOT issue VIRTIO_ADMIN_CMD_LIST_USE commands
if any other command has been submitted to the
device and has not yet completed processing by the device.

The driver SHOULD NOT set bits in device_admin_cmd_opcodes
if it is not familiar with how the command opcode
is used, as dependencies between command opcodes might exist.

The driver MUST NOT request (via VIRTIO_ADMIN_CMD_LIST_USE)
the use of any commands not previously reported as
supported for the same group type by VIRTIO_ADMIN_CMD_LIST_QUERY.

The driver MUST NOT use any commands for a given group type
before sending VIRTIO_ADMIN_CMD_LIST_USE with the correct
list of command opcodes and group type.

The driver MAY block use of VIRTIO_ADMIN_CMD_LIST_QUERY and
VIRTIO_ADMIN_CMD_LIST_USE by issuing VIRTIO_ADMIN_CMD_LIST_USE
with respective bits cleared in \field{command_specific_data}.

The driver MUST handle a command error with a reserved \field{status}
value in the same way as \field{status} set to VIRTIO_ADMIN_STATUS_EINVAL
(except possibly for different error reporting/diagnostic messages).

The driver MUST handle a command error with a reserved
\field{status_qualifier} value in the same way as
\field{status_qualifier} set to
VIRTIO_ADMIN_STATUS_Q_INVALID_COMMAND (except possibly for
different error reporting/diagnostic messages).

When sending commands with the SR-IOV group type,
the driver specify a value for \field{group_member_id}
between $1$ and \field{NumVFs} inclusive,
the driver MUST also make sure that as long as any such command
is outstanding, \field{VF Migration Capable} is clear and
\field{VF Enable} is set;
\field{NumVFs}, \field{VF Migration Capable}  and
\field{VF Enable} refer to registers within the SR-IOV Extended
Capability as specified by \hyperref[intro:PCIe]{[PCIe]}.

\section{Administration Virtqueues}\label{sec:Basic Facilities of a Virtio Device / Administration Virtqueues}

An administration virtqueue of an owner device is used to submit
group administration commands. An owner device can have more
than one administration virtqueue.

If VIRTIO_F_ADMIN_VQ has been negotiated, an owner device exposes one
or more adminstration virtqueues. The number and locations of the
administration virtqueues are exposed by the owner device in a transport
specific manner.

The driver enqueues requests to an arbitrary administration
virtqueue, and they are used by the device on that same
virtqueue. It is the responsibility of the driver to ensure
strict request ordering for commands, because they will be
consumed with no order constraints.  For example, if consistency
is required then the driver can wait for the processing of a
first command by the device to be completed before submitting
another command depending on the first one.

Administration virtqueues are used as follows:
\begin{itemize}
\item The driver submits the command using the \field{struct virtio_admin_cmd}
structure using a buffer consisting of two parts: a device-readable one followed by a
device-writable one.
\item the device-readable part includes fields from \field{opcode}
through \field{command_specific_data}.
\item the device-writeable buffer includes fields from \field{status}
through \field{command_specific_result} inclusive.
\end{itemize}

For each command, this specification describes a distinct
format structure used for \field{command_specific_data} and
\field{command_specific_result}, the length of these fields
depends on the command.

However, to ensure forward compatibility
\begin{itemize}
\item drivers are allowed to submit buffers that are longer
than the device expects
(that is, longer than the length of
\field{opcode} through \field{command_specific_data}).
This allows the driver to maintain
a single format structure even if some structure fields are
unused by the device.
\item drivers are allowed to submit buffers that are shorter
than what the device expects
(that is, shorter than the length of \field{status} through
\field{command_specific_result}). This allows the device to maintain
a single format structure even if some structure fields are
unused by the driver.
\end{itemize}

The device compares the length of each part (device-readable and
device-writeable) of the buffer as submitted by driver to what it
expects and then silently truncates the structures to either the
length submitted by the driver, or the length described in this
specification, whichever is shorter.  The device silently ignores
any data falling outside the shorter of the two lengths. Any
missing fields are interpreted as set to zero.

Similarly, the driver compares the used buffer length
of the buffer to what it expects and then silently
truncates the structure to the used buffer length.
The driver silently ignores any data falling outside
the used buffer length reported by the device.  Any missing
fields are interpreted as set to zero.

This simplifies driver and device implementations since the
driver/device can simply maintain a single large structure (such
as a C structure) for a command and its result. As new versions
of the specification are designed, new fields can be added to the
tail of a structure, with the driver/device using the full
structure without concern for versioning.

\devicenormative{\subsection}{Group administration commands}{Basic Facilities of a Virtio Device / Administration virtqueues}

The device MUST support device-readable and device-writeable buffers
shorter than described in this specification, by
\begin{enumerate}
\item acting as if any data that would be read outside the
device-readable buffers is set to zero, and
\item discarding data that would be written outside the
specified device-writeable buffers.
\end{enumerate}

The device MUST support device-readable and device-writeable buffers
longer than described in this specification, by
\begin{enumerate}
\item ignoring any data in device-readable buffers outside
the expected length, and
\item only writing the expected structure to the device-writeable
buffers, ignoring any extra buffers, and reporting the
actual length of data written, in bytes,
as buffer used length.
\end{enumerate}

The device SHOULD initialize the device-writeable buffer
up to the length of the structure described by this specification or
the length of the buffer supplied by the driver (even if the buffer is
all set to zero), whichever is shorter.

The device MUST NOT fail a command solely because the buffers
provided are shorter or longer than described in this
specification.

The device MUST initialize the device-writeable part of
\field{struct virtio_admin_cmd} that is a multiple of 64 bit in
size.

The device MUST initialize \field{status} and
\field{status_qualifier} in \field{struct virtio_admin_cmd}.

The device MUST process commands on a given administration virtqueue
in the order in which they are queued.

If multiple administration virtqueues have been configured,
device MAY process commands on distinct virtqueues with
no order constraints.

If the device sets \field{status} to either VIRTIO_ADMIN_STATUS_EAGAIN
or VIRTIO_ADMIN_STATUS_ENOMEM, then the command MUST NOT
have any side effects, making it safe to retry.

\drivernormative{\subsection}{Group administration commands}{Basic Facilities of a Virtio Device / Administration virtqueues}

The driver MAY supply device-readable or device-writeable parts
of \field{struct virtio_admin_cmd} that are longer than described in
this specification.

The driver SHOULD supply device-readable part of
\field{struct virtio_admin_cmd} that is at least as
large as the structure described by this specification
(even if the structure is all set to zero).

The driver MUST supply both device-readable or device-writeable parts
of \field{struct virtio_admin_cmd} that are a multiple of 64 bit
in length.

The device MUST supply both device-readable or device-writeable parts
of \field{struct virtio_admin_cmd} that are larger than zero in
length. However, \field{command_specific_data} and
\field{command_specific_result} MAY be zero in length, unless
specified otherwise for the command.

The driver MUST NOT assume that the device will initialize the whole
device-writeable part of \field{struct virtio_admin_cmd} as described in the specification; instead,
the driver MUST act as if the structure
outside the part of the buffer used by the device
is set to zero.

If multiple administration virtqueues have been configured,
the driver MUST ensure ordering for commands
placed on different administration virtqueues.

The driver SHOULD retry a command that completed with
\field{status} VIRTIO_ADMIN_STATUS_EAGAIN.
