\subsubsection{Device and driver capabilities}\label{sec:Basic Facilities of a Virtio Device / Device groups / Group administration commands / Device and driver capabilities}

Device and driver capabilities are implemented as structured groupings for
specific device functionality and their related resource objects. The device exposes
its supported functionality and resource object limits through an administration
command, utilizing the 'self group type.' Each capability possesses a
unique ID. Through an administration command, also employing the
'self group type,' the driver reports the functionality and
resource object limits it intends to use. Before executing any operations
related to the capabilities, the driver communicates these
capabilities to the device. The driver is allowed to set the
capability at any time, provided there are no pending operations
at the device level associated with that capability.

The device presents the supported capability IDs to the driver as a bitmap.
The driver uses the administration command to learn about the
supported capabilities bitmap.

A capability consists of one or more fields, where each field can be a
limit number, a bitmap, or an array of entries. In an array field,
the structure depends on the specific array and the capability type.
For each bitmap field, the driver sets the desired bits - but only out of
those bits in a bitmap that the device has presented.
The driver sets each limit number field to a desired value that
is smaller than or equal to the value the device presented.
Similarly, for an array field, the driver sets the desired capability
entries but only out of the capability entries that the device has presented.

It is anticipated that any necessary new fields for a capability will be
appended to the structure's end, ensuring both forward and backward
compatibility between the device and driver. Furthermore, to avoid
indefinite growth of a single capability, it is expected that new
functionality will lead to the creation of new capability rather
than expanding existing ones.

Capabilities are categorized into two ranges by their IDs, as listed:

\begin{table}[H]
\caption{Capability ids}
\label{table:Basic Facilities of a Virtio Device / Device groups / Group administration commands / Device and driver capabilities / capability ids}
\begin{tabularx}{\textwidth}{ |l|X| }
\hline
Id & Description  \\
\hline \hline
0x0000-0x07ff & Generic capability for all device types \\
\hline
0x0800-0x0fff & Device type specific capability \\
\hline
0x1000 - 0xFFFF & Reserved for future \\
\hline
\end{tabularx}
\end{table}

Common capabilities are listed:

\begin{table}[H]
\caption{Common capability ids}
\label{table:Basic Facilities of a Virtio Device / Device groups / Group administration commands / Device and driver capabilities / Common capability ids}
\begin{tabularx}{\textwidth}{ |l|l|X| }
\hline
Id & Name & Description  \\
\hline \hline
0x0000 & \hyperref[par:Basic Facilities of a Virtio Device / Device groups / Group administration commands / Device parts / VIRTIO-DEV-PARTS-CAP]{VIRTIO_DEV_PARTS_CAP} & Device parts capability \\
\hline
0x0001-0x07ff & - & Generic capability for all device types \\
\hline
\end{tabularx}
\end{table}

Device type specific capabilities are described separately for each device
type under \field{Device and driver capabilities}.

The device and driver capabilities commands are currently defined for self group
type.

\begin{enumerate}
\item VIRTIO_ADMIN_CMD_CAP_ID_LIST_QUERY
\item VIRTIO_ADMIN_CMD_DEVICE_CAP_GET
\item VIRTIO_ADMIN_CMD_DRIVER_CAP_SET
\end{enumerate}

\paragraph{VIRTIO_ADMIN_CMD_CAP_ID_LIST_QUERY}\label{par:Basic Facilities of a Virtio Device / Device groups / Group administration commands / Device and driver capabilities / VIRTIO-ADMIN-CMD-CAP-ID-LIST-QUERY}

This command queries the bitmap of capability ids
listed in \ref{table:Basic Facilities of a Virtio Device / Device groups / Group administration commands / Device and driver capabilities / capability ids}.

For the command VIRTIO_ADMIN_CMD_CAP_ID_LIST_QUERY, \field{opcode} is set to 0x7.
\field{group_member_id} is set to zero.

This command has no command specific data.

\begin{lstlisting}
struct virtio_admin_cmd_query_cap_id_result {
        le64 supported_caps[];
};
\end{lstlisting}

When the command completes successfully, \field{command_specific_result}
is in the format \field{struct virtio_admin_cmd_query_cap_id_result}.

\field{supported_caps} is an array of 64 bit values in little-endian byte
order, in which a bit is set if the specific capability is supported.
Thus, \field{supported_caps[0]} refers to the first 64-bit value in this
array corresponding to capability ids 0 to
63, \field{supported_caps[1]} is the second 64-bit value corresponding to
capability ids 64 to 127, etc. For example, the array of size 2 including
the values 0x3 in \field{supported_caps[0]} and 0x1 in
\field{supported_caps[1]} indicates that only capability id 0, 1
and 64 are supported.
The length of the array depends on the supported capabilities - it is
large enough to include bits set for all supported capability ids,
that is the length can be calculated by starting with the largest
supported capability id adding one, dividing by 64 and rounding up.
In other words, for VIRTIO_ADMIN_CMD_CAP_ID_LIST_QUERY the length of
\field{command_specific_result} will be
$DIV_ROUND_UP(max_cap_id, 64) * 8$ where DIV_ROUND_UP is integer division
with round up and \field{max_cap_id} is the largest available capability id.

The array is also allowed to be larger and to additionally include an arbitrary
number of all-zero entries.

\paragraph{VIRTIO_ADMIN_CMD_DEVICE_CAP_GET}\label{par:Basic Facilities of a Virtio Device / Device groups / Group administration commands / Device and driver capabilities / VIRTIO-ADMIN-CMD-DEVICE-CAP-GET}

This command gets the device capability for the specified capability
id \field{id}.

For the command VIRTIO_ADMIN_CMD_DEVICE_CAP_GET, \field{opcode} is set to 0x8.
\field{group_member_id} is set to zero.

\field{command_specific_data} is in format
\field{struct virtio_admin_cmd_cap_get_data}.

\begin{lstlisting}
struct virtio_admin_cmd_cap_get_data {
        le16 id;
        u8 reserved[6];
};
\end{lstlisting}

\field{id} refers to the capability id listed in \ref{table:Basic Facilities of a Virtio Device / Device groups / Group administration commands / Device and driver capabilities / capability ids}.
\field{reserved} is reserved for future use and set to zero.

\begin{lstlisting}
struct virtio_admin_cmd_cap_get_result {
        u8 cap_specific_data[];
};
\end{lstlisting}

When the command completes successfully, \field{command_specific_result}
is in the format \field{struct virtio_admin_cmd_cap_get_result} responded
by the device. Each capability uses different capability specific
\field{cap_specific_data} and is described separately.

\paragraph{VIRTIO_ADMIN_CMD_DRIVER_CAP_SET}\label{par:Basic Facilities of a Virtio Device / Device groups / Group administration commands / Device and driver capabilities / VIRTIO-ADMIN-CMD-DRIVER-CAP-SET}

This command sets the driver capability, indicating to the device which capability
the driver uses. The driver can set a resource object limit capability that is smaller
than or equal to the value published by the device capability.
If the capability is a set of flags, the driver sets the flag bits that are set
in the device capability; the driver does not set any flag bits that are not
set by the device.

For the command VIRTIO_ADMIN_CMD_DRIVER_CAP_SET, \field{opcode} is set to 0x9.
\field{group_member_id} is set to zero.
The \field{command_specific_data} is in the format
\field{struct virtio_admin_cmd_cap_set_data}.

\begin{lstlisting}
struct virtio_admin_cmd_cap_set_data {
        le16 id;
        u8 reserved[6];
        u8 cap_specific_data[];
};
\end{lstlisting}

\field{id} refers to the capability id listed in \ref{table:Basic Facilities of a Virtio Device / Device groups / Group administration commands / Device and driver capabilities / capability ids}.
\field{reserved} is reserved for future use and set to zero.

There is no command specific result.
When the command completes successfully, the driver capability is updated to
the values supplied in \field{cap_specific_data}.

\devicenormative{\paragraph}{Device and driver capabilities}{Basic Facilities of a Virtio Device / Device groups / Group administration commands / Device and driver capabilities}

If the device supports capabilities, it MUST support the commands
VIRTIO_ADMIN_CMD_CAP_ID_LIST_QUERY,
VIRTIO_ADMIN_CMD_DRIVER_CAP_SET, and
VIRTIO_ADMIN_CMD_DEVICE_CAP_GET.

For the VIRTIO_ADMIN_CMD_DRIVER_CAP_SET command,
\begin{itemize}
\item the device MUST support the setting of resource object limit driver capability to a
value that is same as or smaller than the one reported in the device
capability,
\item the device MUST support the setting of capability flags bits to
all or fewer bits than the one reported in the device capability;
\end{itemize}
this is applicable unless specific capability fields are explicitly
stated as non-writable in the VIRTIO_ADMIN_CMD_DEVICE_CAP_GET command.

The device MAY complete the command VIRTIO_ADMIN_CMD_DRIVER_CAP_SET with
\field{status} set to VIRTIO_ADMIN_STATUS_EINVAL,
if the capability resource object limit is larger than the value reported by the
device's capability, or the capability flag bit is set, which is not set in
the device's capability.

The device MUST complete the commands VIRTIO_ADMIN_CMD_CAP_ID_LIST_QUERY,
VIRTIO_ADMIN_CMD_DRIVER_CAP_GET, and VIRTIO_ADMIN_CMD_DRIVER_CAP_SET
with \field{status} set to VIRTIO_ADMIN_STATUS_EINVAL if the commands are not
for the self group type.

The device SHOULD complete the commands VIRTIO_ADMIN_CMD_CAP_ID_LIST_QUERY,
VIRTIO_ADMIN_CMD_DRIVER_CAP_GET, VIRTIO_ADMIN_CMD_DRIVER_CAP_SET with
\field{status} set to VIRTIO_ADMIN_STATUS_EINVAL if the commands are not for the
self group type.

The device SHOULD complete the command VIRTIO_ADMIN_CMD_DRIVER_CAP_SET with
\field{status} set to VIRTIO_ADMIN_STATUS_EBUSY if the command requests to disable
a capability while the device still has valid resource objects related to the
capability being disabled.

The device SHOULD complete the comands VIRTIO_ADMIN_CMD_DEVICE_CAP_GET and
VIRTIO_ADMIN_CMD_DRIVER_CAP_SET with \field{status} set to
VIRTIO_ADMIN_STATUS_ENXIO if the capability id is not reported
in command VIRTIO_ADMIN_CMD_CAP_ID_LIST_QUERY.

Upon a device reset, the device MUST reset all driver capabilities.

The device SHOULD treat the driver resource limits as zero if the
driver has not set such capability, unless otherwise explicitly stated.

\drivernormative{\paragraph}{Device and driver capabilities}{Basic Facilities of a Virtio Device / Device groups / Group administration commands / Device and driver capabilities}

The driver MUST send the command VIRTIO_ADMIN_CMD_DRIVER_CAP_SET before
using any resource objects that depend on such a capability.

In VIRTIO_ADMIN_CMD_DRIVER_CAP_SET command, the driver MUST NOT set
\begin{itemize}
\item the resource object limit value larger than the value reported
by the device in the command VIRTIO_ADMIN_CMD_DEVICE_CAP_GET,
\item flags bits which was not reported by the device in the command
VIRTIO_ADMIN_CMD_DEVICE_CAP_GET,
\item array entries not reported by the device in the command
VIRTIO_ADMIN_CMD_DEVICE_CAP_GET.
\end{itemize}

The driver MUST NOT disable any of the driver capability using the command
VIRTIO_ADMIN_CMD_DRIVER_CAP_SET when related resource objects
are created but not destroyed.
