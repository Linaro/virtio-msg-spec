\subsubsection{Device resource objects}\label{sec:Basic Facilities of a Virtio Device / Device groups / Group administration commands / Device resource objects}

Providing certain functionality consumes limited device resources such as
memory, processing units, buffer memory, or end-to-end credits. A device may
support multiple types of resource objects, each controlling different device
functionality. To manage this, virtio provides
\field{Device resource objects} that the driver can create, modify, and
destroy using administration commands with the self group type. Creating and
destroying a resource object consume and release device resources, respectively.
The device resource object query command returns the resource object as
maintained by the device.

For each resource type, the number of resource objects that can be created
is reported by the device as part of a device capability
\ref{sec:Basic Facilities of a Virtio Device / Device groups / Group administration commands / Device and driver capabilities}.
The driver reports the desired (same or lower) number of resource objects
as part of a driver capability \ref{sec:Basic Facilities of a Virtio Device / Device groups / Group administration commands / Device and driver capabilities}.
For each device object type, resource object limit is defined by field
\field{limit} using \field{Device and driver capabilities}.

\begin{lstlisting}
le32 limit; /* maximum resource id = limit - 1 */
\end{lstlisting}

Each resource object has a unique resource object ID - a driver-assigned number
in the range of 0 to \field{limit - 1}, where the \field{limit} is the maximum
number set by the driver for this resource object type. These resource IDs are unique within
each resource object type. The driver assigns the resource ID when creating a
device resource object. Once the resource object is successfully created,
subsequent resource modification, query, and destroy commands use this
resource object ID. No two resource objects share the same ID. Destroying a
resource object allows for the reuse of its ID for another resource object
of the same type.

A valid resource object id is \field{limit - 1}. For example, when a device
reports a \field{limit = 10} capability for a resource object, and drivers sets
\field{limit = 8}, the valid resource object id range for the device and the
driver is 0 to 7 for all the resource object commands. In this example,
the driver can only create 8 resource objects of a specified type.

A resource object of one type may depend on the resource object of another type.
Such dependency between resource objects is established by referring to the unique
resource ID in the administration commands. For example, a driver creates a
resource object identified by ID A of one type, then creates another resource
object identified by ID B of a different type, which depends on resource object
A. This dependency establishes the lifecycle of these resource objects. The driver
that creates the dependent resource object must destroy the resource objects in the
exact reverse order of their creation. In this example, the driver would
destroy resource object B before destroying resource object A.

Some resource object types are generic, common across multiple devices.
Others are specific for one device type.

\begin{tabular}{|l|l|}
\hline
Resource object type & Description \\
\hline \hline
0x000-0x1ff & Generic resource object type common across all devices \\
\hline
0x200-0x4ff & Device type specific resource object \\
\hline
0x500-0xffff & Reserved for future use  \\
\hline
\end{tabular}

When the device resets, it starts with zero resources of each type; the driver
can create resources up to the published \field{limit}. The driver can
destroy and recreate the resource one or multiple times. Upon device reset,
all resource objects created by the driver are destroyed within the device.
