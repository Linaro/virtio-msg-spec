\subsubsection{Device resource objects}\label{sec:Basic Facilities of a Virtio Device / Device groups / Group administration commands / Device resource objects}

Providing certain functionality consumes limited device resources such as
memory, processing units, buffer memory, or end-to-end credits. A device may
support multiple types of resource objects, each controlling different device
functionality. To manage this, virtio provides
\field{Device resource objects} that the driver can create, modify, and
destroy using administration commands with the self group type. Creating and
destroying a resource object consume and release device resources, respectively.
The device resource object query command returns the resource object as
maintained by the device.

For each resource type, the number of resource objects that can be created
is reported by the device as part of a device capability
\ref{sec:Basic Facilities of a Virtio Device / Device groups / Group administration commands / Device and driver capabilities}.
The driver reports the desired (same or lower) number of resource objects
as part of a driver capability \ref{sec:Basic Facilities of a Virtio Device / Device groups / Group administration commands / Device and driver capabilities}.
For each device object type, resource object limit is defined by field
\field{limit} using \field{Device and driver capabilities}.

\begin{lstlisting}
le32 limit; /* maximum resource id = limit - 1 */
\end{lstlisting}

Each resource object has a unique resource object ID - a driver-assigned number
in the range of 0 to \field{limit - 1}, where the \field{limit} is the maximum
number set by the driver for this resource object type. These resource IDs are unique within
each resource object type. The driver assigns the resource ID when creating a
device resource object. Once the resource object is successfully created,
subsequent resource modification, query, and destroy commands use this
resource object ID. No two resource objects share the same ID. Destroying a
resource object allows for the reuse of its ID for another resource object
of the same type.

A valid resource object id is \field{limit - 1}. For example, when a device
reports a \field{limit = 10} capability for a resource object, and drivers sets
\field{limit = 8}, the valid resource object id range for the device and the
driver is 0 to 7 for all the resource object commands. In this example,
the driver can only create 8 resource objects of a specified type.

A resource object of one type may depend on the resource object of another type.
Such dependency between resource objects is established by referring to the unique
resource ID in the administration commands. For example, a driver creates a
resource object identified by ID A of one type, then creates another resource
object identified by ID B of a different type, which depends on resource object
A. This dependency establishes the lifecycle of these resource objects. The driver
that creates the dependent resource object must destroy the resource objects in the
exact reverse order of their creation. In this example, the driver would
destroy resource object B before destroying resource object A.

Some resource object types are generic, common across multiple devices.
Others are specific for one device type.

\begin{tabular}{|l|l|}
\hline
Resource object type & Description \\
\hline \hline
0x000-0x1ff & Generic resource object type common across all devices \\
\hline
0x200-0x4ff & Device type specific resource object \\
\hline
0x500-0xffff & Reserved for future use  \\
\hline
\end{tabular}

When the device resets, it starts with zero resources of each type; the driver
can create resources up to the published \field{limit}. The driver can
destroy and recreate the resource one or multiple times. Upon device reset,
all resource objects created by the driver are destroyed within the device.

Following administration commands control device resource objects,
they are only supported for the self group type:

\begin{enumerate}
\item VIRTIO_ADMIN_CMD_RESOURCE_OBJ_CREATE
\item VIRTIO_ADMIN_CMD_RESOURCE_OBJ_MODIFY
\item VIRTIO_ADMIN_CMD_RESOURCE_OBJ_QUERY
\item VIRTIO_ADMIN_CMD_RESOURCE_OBJ_DESTROY
\end{enumerate}

Each resource object administration command uses a common header
\field{struct virtio_admin_cmd_resource_obj_cmd_hdr}.

\begin{lstlisting}
struct virtio_admin_cmd_resource_obj_cmd_hdr {
        le16 type;
        u8 reserved[2];
        le32 id; /* Indicates unique resource object id per resource object type */
};

\end{lstlisting}

\field{type} refers to the device resource object type.
\field{id} uniquely identifies the resource object of a specified \field{type}.

\paragraph{VIRTIO_ADMIN_CMD_RESOURCE_OBJ_CREATE}
\label{par:Basic Facilities of a Virtio Device / Device groups / Group administration commands / Device resource objects / VIRTIO_ADMIN_CMD_RESOURCE_OBJ_CREATE}

This command creates the specified resource object of \field{type} identified by the
resource id \field{id}. The valid range of \field{id} is defined by the
device in the related device capability. The driver assigns the unique \field{id}
for the resource for the specified \field{type}.

For the command VIRTIO_ADMIN_CMD_RESOURCE_OBJ_CREATE, \field{opcode} is set to 0xa.
\field{group_member_id} is set to zero.
The \field{command_specific_data} is in the format
\field{struct virtio_admin_cmd_resource_obj_create_data}.
\field{resource_obj_specific_data} refers to the resource object specific data.
Each resource uses a different \field{resource_obj_specific_data} and is described
separately.

\field{flags} is reserved for future extension for optional resource object attributes and
is set to 0. Each resource object uses a different value for
\field{flags} and it is described separately.

\begin{lstlisting}
struct virtio_admin_cmd_resource_obj_create_data {
        struct virtio_admin_cmd_resource_obj_cmd_hdr hdr;
        le64 flags;
        u8 resource_obj_specific_data[];
};
\end{lstlisting}

When the command completes successfully, the resource object is created by the
device and the device can immediately begin using it.
This command has no command specific result.

\paragraph{VIRTIO_ADMIN_CMD_RESOURCE_OBJ_MODIFY}
\label{par:Basic Facilities of a Virtio Device / Device groups / Group administration commands / Device resource objects / VIRTIO_ADMIN_CMD_RESOURCE_OBJ_MODIFY}

This command modifies the attributes of an existing device resource object.
For the command VIRTIO_ADMIN_CMD_RESOURCE_OBJ_MODIFY, \field{opcode} is set to 0xb.
The \field{command_specific_data} is in the format
\field{struct virtio_admin_cmd_resource_modify_data}.
\field{group_member_id} is set to zero.
\field{id} identifies the resource object of type \field{type} whose attributes
to modify.
This command modifies the attributes supplied in \field{resource_obj_specific_data}.

\begin{lstlisting}
struct virtio_admin_cmd_resource_modify_data {
        struct virtio_admin_cmd_resource_obj_cmd_hdr hdr;
        le64 flags;
        u8 resource_obj_specific_data[];
};
\end{lstlisting}

This command has no command specific result.
When the command completes successfully, attributes of the resource object is
set to the values supplied in \field{resource_obj_specific_data}.

\paragraph{VIRTIO_ADMIN_CMD_RESOURCE_OBJ_QUERY}
\label{par:Basic Facilities of a Virtio Device / Device groups / Group administration commands / Device resource objects / VIRTIO_ADMIN_CMD_RESOURCE_OBJ_QUERY}

This command queries attributes of the existing resource object.
For the command VIRTIO_ADMIN_CMD_RESOURCE_OBJ_QUERY, \field{opcode} is set to 0xc.
\field{group_member_id} is set to zero.
The \field{command_specific_data} is in the format
\field{struct virtio_admin_cmd_resource_obj_query_data}.
\field{id} identifies the existing resource object of type \field{type} whose
attributes to query.

\begin{lstlisting}
struct virtio_admin_cmd_resource_obj_query_data {
        struct virtio_admin_cmd_resource_obj_cmd_hdr hdr;
        le64 flags;
};
\end{lstlisting}

\begin{lstlisting}
struct virtio_admin_cmd_resource_obj_query_result {
        u8 resource_obj_specific_result[];
};
\end{lstlisting}

\field{command_specific_result} is in the format
\field{virtio_admin_cmd_resource_obj_query_result}.

When the command completes successfully, the attributes of the specified
resource object are are set in \field{resource_obj_specific_data}.

\paragraph{VIRTIO_ADMIN_CMD_RESOURCE_OBJ_DESTROY}
\label{par:Basic Facilities of a Virtio Device / Device groups / Group administration commands / Device resource objects / VIRTIO_ADMIN_CMD_RESOURCE_OBJ_DESTROY}

This command destroys the previously created device resource object.
For the command VIRTIO_ADMIN_CMD_RESOURCE_OBJ_DESTROY, \field{opcode} is set to 0xd.
The \field{command_specific_data} is in the format
\field{struct virtio_admin_cmd_resource_obj_cmd_hdr}.
\field{group_member_id} is set to zero.
\field{id} identifies the existing resource object of type \field{type}.

This command destroys the specified resource object of \field{type} identified
by \field{id}, which is previously created using
VIRTIO_ADMIN_CMD_RESOURCE_OBJ_CREATE command.

This command has no command specific result.
When the command completes successfully, the resource object is destroyed from the device.

\devicenormative{\paragraph}{Device resource objects}{Basic Facilities of a Virtio Device / Device groups / Group administration commands / Device resource objects}

The device SHOULD complete the command VIRTIO_ADMIN_CMD_RESOURCE_OBJ_CREATE
with \field{status} set to VIRTIO_ADMIN_STATUS_EEXIST if a resource object already exists
with supplied resource \field{id} for the specified \field{type}.

The device SHOULD complete the commands VIRTIO_ADMIN_CMD_RESOURCE_OBJ_MODIFY,
VIRTIO_ADMIN_CMD_RESOURCE_QUERY and
VIRTIO_ADMIN_CMD_RESOURCE_OBJ_DESTROY with \field{status} set to
VIRTIO_ADMIN_STATUS_ENXIO if the specified resource object does not exist.

The device SHOULD set \field{status} to VIRTIO_ADMIN_STATUS_ENOSPC for the
command VIRTIO_ADMIN_CMD_RESOURCE_OBJ_CREATE if the device fail to create the
resource object.

The device SHOULD complete the commands VIRTIO_ADMIN_CMD_RESOURCE_OBJ_MODIFY or
VIRTIO_ADMIN_CMD_RESOURCE_OBJ_DESTROY commands with \field{status} set to
VIRTIO_ADMIN_STATUS_EBUSY if other resource objects depend on the resource object
being modified or destroyed.

The device MUST allow recreating the resource object using the command
VIRTIO_ADMIN_CMD_RESOURCE_OBJ_CREATE which was previously
destroyed using the command VIRTIO_ADMIN_CMD_RESOURCE_OBJ_DESTROY respectively
without undergoing a device reset.

The device SHOULD allow creating the resource object using
the command VIRTIO_ADMIN_CMD_RESOURCE_OBJ_CREATE with any resource
id as long as the resource object is not created.

The device MAY fail the command VIRTIO_ADMIN_CMD_RESOURCE_OBJ_CREATE even if the
resources within the device have not reached up to the \field{max_limit}
but the device MAY have reached an internal limit.

When a capability represents a number of resource objects, the device SHOULD
allow creating as many resource objects as represented by the driver capability.

The device MUST NOT have any side effects on the resource object when the command
VIRTIO_ADMIN_CMD_RESOURCE_OBJ_MODIFY fails.

The device MUST complete the command VIRTIO_ADMIN_CMD_RESOURCE_OBJ_QUERY
with \field{resource_obj_specific_data} which is matching the
\field{resource_obj_specific_data} of last VIRTIO_ADMIN_CMD_RESOURCE_OBJ_CREATE
or VIRTIO_ADMIN_CMD_RESOURCE_OBJ_MODIFY command.

On device reset, the device MUST destroy all the resource objects which
have been created.

\drivernormative{\paragraph}{Device resource objects}{Basic Facilities of a Virtio Device / Device groups / Group administration commands / Device resource objects}

The driver MUST not create a second resource object of the same type with same
ID using command VIRTIO_ADMIN_CMD_RESOURCE_OBJ_CREATE before destroying the
previously created resource object.

The driver MUST NOT create more resource objects of a specified \field{type} using
command VIRTIO_ADMIN_CMD_RESOURCE_OBJ_CREATE than the maximum limit set by the
driver capability.

The driver SHOULD NOT modify, query and destroy the resource object which is
already destroyed previously by the driver.

The driver SHOULD NOT destroy the resource object on which other resource objects
are depending; the driver SHOULD destroy all the resource objects which do not depend
on other resource objects.

The driver MUST NOT set the capability related to the resource objects if the
resource objects have been created using the command VIRTIO_ADMIN_CMD_RESOURCE_OBJ_CREATE
and not yet destroyed.

The driver MUST send the command VIRTIO_ADMIN_CMD_DRIVER_CAP_SET before using
any resources related to such capability.
